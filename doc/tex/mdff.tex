%MDFF documentation
\documentclass[a4paper,8pt]{article}
\usepackage{lscape}
\usepackage{geometry}
\usepackage[strict]{changepage}
\usepackage[utf8]{inputenc}
\usepackage[T1]{fontenc}
\usepackage{multirow}
\usepackage{array}
\usepackage{longtable}
\usepackage{color}
\usepackage{amssymb}
\usepackage{amsmath}


%##############################################
%new commands
%##############################################
\newcommand{\MDFF}{{\sc MDff}}
\newcommand{\FMV}{{\sc fmv}}
\newcommand{\wiki}{\textcolor{red}{[wiki]}} 
\newcommand{\addref}{\textcolor{blue}{[ref]}}

% math symbol
\def\nn{{\bf n}}
\def\kk{{\bf k}}
\def\rr{{\bf r}}
\def\00{{\bf 0}}




\title{Molecular Dynamics FF, Manual }

\begin{document}

%\changepage{ text height }{ text width }{ even-side margin } { odd-side margin }{ column sep. }{ topmargin }{ headheight }{ headsep }{ footskip }
\changepage{0pt}{30pt}{0pt}{0pt}{0pt}{0pt}{0pt}{0pt}{0pt}
\maketitle
\clearpage
\tableofcontents

\clearpage

\section{Aim}

Le but de ce document est de repertorier les differents 
algorithmes utilis\'es par \MDFF. Notamment, definir les 
commandes (TAG) et leurs valeurs par défaut. Mais également, 
les r\'eferences aux codes, articles originaux, livres dont 
sont extraits les diff\'erents algorithmes, ou preciser dans 
le cas contraire (i.e \FMV!) ce que fait la routine point par point. 
Ce travail de documentation permettra de faciliter la d\'ecouverte
de bug qui independamment de ma volont\'e sont (et seront) 
forc\'ement pr\'esents. 

\clearpage

\section{Theory}

%%%%%%%%%%%%%%%%%%%%%%%%%%%%%%%%%%%%%%%%%%%%%%%%%%%%%%%%%%%%
\subsection{Molecular Dynamics\label{sec:MD}}
%%%%%%%%%%%%%%%%%%%%%%%%%%%%%%%%%%%%%%%%%%%%%%%%%%%%%%%%%%%%
\subsubsection{Integration\label{sec:integrator}}
\subsubsection{Force-Field}
\paragraph{Binary Mixture Lennard-Jones\label{sec:BMLJ}}
\paragraph{Kob-Andersen model\label{sec:KA}}
\paragraph{Morse potential\label{sec:morse}}
\paragraph{Born-Mayer-Huggings-Fumi-Tosi\label{sec:BMHFT}}
\paragraph{Long-range electrostatic\label{sec:COUL}}
\paragraph{Multipole expansion: polarizabilities \label{sec:pola}}


%%%%%%%%%%%%%%%%%%%%%%%%%%%%%%%%%%%%%%%%%%%%%%%%%%%%%%%%%%%%
\subsection{Optimisation\label{sec:OPT}}
%%%%%%%%%%%%%%%%%%%%%%%%%%%%%%%%%%%%%%%%%%%%%%%%%%%%%%%%%%%%

%%%%%%%%%%%%%%%%%%%%%%%%%%%%%%%%%%%%%%%%%%%%%%%%%%%%%%%%%%%%
\subsection{Phonons\label{sec:VIB}}
%%%%%%%%%%%%%%%%%%%%%%%%%%%%%%%%%%%%%%%%%%%%%%%%%%%%%%%%%%%%

%%%%%%%%%%%%%%%%%%%%%%%%%%%%%%%%%%%%%%%%%%%%%%%%%%%%%%%%%%%%
\subsection{Electric-Field Gradient\label{sec:EFG}}
%%%%%%%%%%%%%%%%%%%%%%%%%%%%%%%%%%%%%%%%%%%%%%%%%%%%%%%%%%%%

\subsubsection{Some general definition: NMR oriented}

In solid-state NMR, the  quadrupolar interaction is  characterised by two parameters:
$C_Q$ and $\eta_Q$ (respectively the quadrupolar coupling constant, and  the
quadrupolar asymmetry), which are both defined from the  three principal  components
of the diagonal  EFG tensor  ($\bf{V}$), by relations:
\begin{equation}
  \label{eq:quad}
 C_Q=\frac{e|Q|}{h}|V_{zz}| ,\;\;\;\;\;\; \eta_Q=\frac{V_{yy}-V_{xx}}{V_{zz}}
\end{equation}
where $e$ is the elementary charge, $h$ is the Planck's constant, $Q$
is the quadrupolar  moment and the principal components $V_{ii},\ (i=x,y,z)$ are sorted
such that  $|V_{zz}|  \geqslant |V_{xx}| \geqslant |V_{yy}|$. 
We only consider the parameters $V_{zz}$ and $\eta_Q$ to report the principal
components of the diagonal EFG tensor. Even if, NMR experiments at room temperature, are unable
to determined the sign of the larger principal component~\cite{B-Abragam}, this sign property,
which is explicitly available by DFT calculation, is important to characterise the distribution~\cite{VasconcelosPRB}

In the general  case, the  zero trace  EFG tensor  is not
diagonal,  and it  is  necessary to  defined  three others  parameters
which  define the orientation  of the tensor in a fixed reference frame.
Consequently, the EFG tensor is completely defined by five independent
variables as expected by a symmetric second-rank tensor without trace.
Following Czjzek at al.~\cite{PRB.23.2513}, we define five real parameters
$U_i$ from the Cartesian components of the EFG tensor.
\begin{equation}
  \label{eq:U_i}
  U_1=v_{zz}/2, U_2=\frac{v_{xz}}{\sqrt{3}},U_3=\frac{v_{yz}}{\sqrt{3}},U_4=\frac{v_{xy}}{\sqrt{3}},U_5=\frac{(v_{xx}-v_{yy})}{2\sqrt{3}}  
\end{equation}
The five-dimensional vector ${\bf{U}}=(U_1,\ldots,U_5)$ constitutes a random vector
representative of the EFG tensor distribution of an amorphous solid in a fixed frame
of reference for all the nuclei sites. To get access to the distribution of the two
parameters $V_{zz}$ and $\eta_Q$, it is necessary to diagonalise the EFG tensor from
site to site~\cite{JPCM.10.10715}.


\subsubsection{EFG calculation: Implemented formulas}

Consider a finite number $N$ of point charge particules, the EFG tensor
($\bf{V}$) at the position $\rr_i$ of a particule $i$ due to the
$N-1$ other point charges $q_j$ at distance $\rr_{ij}=\rr_i-\rr_j$.
\begin{equation}
V_{i,\alpha\beta} = - \sum_{j\ne i}^N q_j \bigg( \frac {3 r_{ij,\alpha} r_{ij,\beta} - r_{ij}^2\delta_{\alpha\beta} } {r_{ij}^5} \bigg )
\label{eq:EFG}
\end{equation}
where  $\alpha$ end $\beta$ are the cartesian components, $\delta_{\alpha\beta}$
is the Kronecker delta.
In periodic boundaries conditions, a further summation is added to
Eq.~\ref{eq:EFG} arising from the particules which belong to other cells.
This method is called the direct summation. The EFG tensor by the direct
summation technique is given by
%\begin{widetext}
\begin{equation}
V^{DS}_{i,\alpha\beta} = - \sum_\nn^{n_{max}} \sideset{}{'}\sum^{N}_j \\ q_j\bigg(\frac{3 (r_{ij,\alpha}+n_\alpha) (r_{ij,\beta}+n_\beta) - |\rr_{ij}+\nn|^2\delta_{\alpha\beta} } { |\rr_{ij}+\nn|^5 } \bigg )
\end{equation}
%\end{widetext}
where $\nn$ is an integer vector such $\nn=(n_x,n_y,n_z)L$, the prime indicates
that the $j=i$ term is ommited when $\nn=\00$. In practice a further spherical
radial cutoff $r^{cut}_{efg}$ is applied to the cubic cutoff $n_{max}$ which is set
according to the ratio $r^{cut}_{efg}/L$.

%As the EFG scales as $r^{-3}$, the direct summation technique converge relatively
%fast. For a typical configurations discussed in that paper, we oberved that the EFG
%tensor converged within a spherical radius cutoff of 40 $[\sigma_{AA}]$,
%which corresponds to $n_{max}=$~4.

The EFG tensor can be also obtained from an Ewald summation
~\cite{ANDP:ANDP19213690304,B-ALLEN_TILDESLEY} like technique.
The main advantage of this approach is the rapid convergence of
the k-space summation compared to the direct summation only technique.
%Ici on suit la notation de Nymand and Linse(en partie), mais les equations
% correspondent aux equations implementees dans MDFF. 
The EFG tensor of atom $i$ from the Ewald Sum $V^{ES}_{i,\alpha\beta}$ is
\begin{equation}
V^{ES}_{i,\alpha\beta} = V^{real}_{i,\alpha\beta} + V^{rec}_{i,\alpha\beta}
\label{eq:EFG_ES}
\end{equation}
where $V^{real}_{i,\alpha\beta}$ is the real part contribution given by
\begin{equation}
V^{real}_{i,\alpha\beta} = - \sum^{N}_{j\ne i} q_j (3 r_{ij,\alpha}r_{ij,\beta} - r_{ij}^2\delta_{\alpha\beta} ) ( T_0 + T_1 + T_2 ) 
\end{equation}
with
\begin{eqnarray}
T_0&=& \frac{{\rm erfc}( \alpha r_{ij} )} {r_{ij}^5} \nonumber \\
T_1&=& \frac{2 \alpha \exp(-\alpha^2 r_{ij}^2)}{\sqrt{\pi}r_{ij}^4}  \nonumber \\
T_2&=& \frac{4 \alpha^3 \exp(-\alpha^2 r_{ij}^2)}{3\sqrt{\pi} r_{ij}^2};
\end{eqnarray}
and where $V^{rec}_{i,\alpha\beta}$ is the reciprocal contribution given by
\begin{equation}
V^{rec}_{i,\alpha\beta} = \frac{4\pi}{3V}\sum_{\kk\ne\00} A_k Re\Big[ \sum^N_j q_j e^{i\kk\cdot(\rr_i-\rr_j)} \Big] \Big( 3 k_\alpha k_\beta - k^2\delta_{\alpha\beta} \Big)
\end{equation}
with
\begin{equation}
A_k = \frac{\exp( - k^2 / 4\alpha^2 )}{k^2} 
\end{equation}
In practice, we calculate the charge density $\rho_N(\kk)$ defined in k-space by
\begin{equation}
\rho_N(\kk) = \sum^N_j q_j e^{i\kk\rr_j}
\end{equation}
only once at the beginning, and for each $i$ particule and k-point mesh
we calculate $e^{i\kk\rr_i} \rho^*_N(\kk)$ where $*$ denotes the complex conjugate.
The k-space summation runs over all k-points defined by the vector
\begin{eqnarray}
\kk=2\pi(n_x/L,n_y/L,n_z/L) \\
{\rm with}\; |n_x|,|n_y|,|n_z| < n_{max}.
\end{eqnarray}
The convergence of the EFG obtained by the Ewald Summation technique
depends on two main parameters, the Ewald screening parameter $\alpha$
and on the k-space cut-off $k_{max}=2\pi n_{max}/L$.

In the direct summation procedure, there is no need to use the minimum
 image convention as all the neighbour cells are explicitely take into account.
In the real part of the Ewald summation, however, the minimum image convention
is used. the screening parameter $\alpha$ should choose such that that only particles
in the primary box need to be considered when calculating the real space sum.
This cutoff is then implictely set to half of the box size.


%%%%%%%%%%%%%%%%%%%%%%%%%%%%%%%%%%%%%%%%%%%%%%%%%%%%%%%%%%%%
\subsection{Radial distribution function\label{sec:GR}}
%%%%%%%%%%%%%%%%%%%%%%%%%%%%%%%%%%%%%%%%%%%%%%%%%%%%%%%%%%%%

%%%%%%%%%%%%%%%%%%%%%%%%%%%%%%%%%%%%%%%%%%%%%%%%%%%%%%%%%%%%
\subsection{Verlet list\label{sec:vnlist}}
%%%%%%%%%%%%%%%%%%%%%%%%%%%%%%%%%%%%%%%%%%%%%%%%%%%%%%%%%%%%

%%%%%%%%%%%%%%%%%%%%%%%%%%%%%%%%%%%%%%%%%%%%%%%%%%%%%%%%%%%%
\subsection{Units \label{sec:units}}
%%%%%%%%%%%%%%%%%%%%%%%%%%%%%%%%%%%%%%%%%%%%%%%%%%%%%%%%%%%%

%%%%%%%%%%%%%%%%%%%%%%%%%%%%%%%%%%%%%%%%%%%%%%%%%%%%%%%%%%%%
\subsection{Miscellaneous \label{sec:misc}}
%%%%%%%%%%%%%%%%%%%%%%%%%%%%%%%%%%%%%%%%%%%%%%%%%%%%%%%%%%%%

\section{Usage}

The main file (usually called control.F~\footnote{the .F extension permits to get nice colors with vim}) is read in input as :
\begin{verbatim}
mdff.x control.F
\end{verbatim}

\MDFF~reads configuration from different files, depending on the task :
\begin{verbatim}
POSFF : md
TRAJFF : opt, efg
ISCFF : vib
\end{verbatim}

\subsection{Main input : control.F}

The input file is composed of different sections. 
Each section is defined by  \&SECTIONNAME and \&END.
Everything outside the sections is ignored.
\\
\\
In \MDFF, three sections are always required :

\begin{itemize}
\item \&controltag ... \&end $\rightarrow$ General control parameters for calculation 
\item \&mdtag ... \&end      $\rightarrow$ General Molecular dynamics parameters.
\item \&fieldtag ... \&end   $\rightarrow$ Set the Force-Field.
\end{itemize}

see examples/ex0/control.F for the some minimum settings example
Using this minimum settings the code runs with several default values.
It correspond to a molecular dynamics trajectory of 10 unit of times of a fcc lennard-jones 
crystal (256 atoms) at $\rho=$1.0 at $T=$1.0. In the next sextion we show the standard output of this minimum setting. 

\clearpage
\subsection{Minimum setting: standard output}
\begin{verbatim}
                            \\|//                            
                           -(o o)-                           
========================oOO==(_)==OOo========================
          ____    ____  ______   ________  ________  
         |_   \  /   _||_   _ `.|_   __  ||_   __  | 
           |   \/   |    | | `. \ | |_ \_|  | |_ \_| 
           | |\  /| |    | |  | | |  _|     |  _|    
          _| |_\/_| |_  _| |_.' /_| |_     _| |_     
         |_____||_____||______.'|_____|   |_____|    

=============================================================

MOLECULAR DYNAMICS ...for fun                 
mdff.r70  (build Mar 28 2014 12:40:00) parallel                                 
filipe.manuel.vasconcelos@gmail.com  
Running on  :    1 nodes                  
by user     : filipe                        
host        : IRAM-CE-004776
date        : 2014/03/28    12:45:23

CONTROL MODULE ... WELCOME

calc        =  md                                                          
lnmlj       =  T
lbmhft      =  F
lbmhftd     =  F
lmorse      =  F
lcoulomb    =  F
lsurf       =  F
lvnlist     =  T
lstatic     =  T
lreduced    =  F
lrestart    =  F

=============================================================

reading configuration
config from file POSFF
found type information on POSFF : A  
atomic positions in direct coordinates in POSFF

=============================================================

CONFIG MODULE ... WELCOME

system                : CONFIG                                                      
natm                  =             4000
ntype                 =                1
nA                    =             4000  100.00%

-------------------------------------------------------------
direct     basis : 
-------------------------------------------------------------
a_vector              =      15.4144      0.0000      0.0000
b_vector              =       0.0000     15.4144      0.0000
c_vector              =       0.0000      0.0000     15.4144
cell param.           =      15.4144     15.4144     15.4144
perpend. width        =      15.4144     15.4144     15.4144
angles                =      90.0000     90.0000     90.0000
volume                =    3662.4913


-------------------------------------------------------------
reciprocal basis : 
-------------------------------------------------------------
a*_vector             =       0.0649      0.0000      0.0000
b*_vector             =       0.0000      0.0649      0.0000
c*_vector             =       0.0000      0.0000      0.0649

cell param.           =       0.0649      0.0649      0.0649
perpend. width        =       0.0649      0.0649      0.0649
angles                =      90.0000     90.0000     90.0000
volume                =       1.0000

-------------------------------------------------------------

density               =       1.0922

 long range correction (init)  -169.435784080640     
=============================================================

FIELD MODULE ... WELCOME

-------------------------------------------------------------
point charges: 
-------------------------------------------------------------
q   A                   =  0.000E+00
quadA                   =  0.000E+00 mb
total charge            =  0.000E+00
second moment of charge =  0.000E+00

-------------------------------------------------------------
static dipoles: 
-------------------------------------------------------------
muA        =    0.00000   0.00000   0.00000

-------------------------------------------------------------
lennard-jones           
-------------------------------------------------------------

       eps    /    / sigma* \ q       / sigma*  \ p \
 V = ------- |  p | ------- |    - q | -------- |   |
      q - p   \    \   r    /         \    r    /   /

cutoff      =    6.00000
truncation  = linear                                                      
long range correction :       -169.435784081


-------------------------------------------------------------
A  -A   interactions:
-------------------------------------------------------------
sigma                                =   0.10000000E+01
eps                                  =   0.10000000E+01
q                                    =   0.12000000E+02
p                                    =   0.60000000E+01
shift correction (linear)            :   -0.00009


=============================================================

=============================================================

MD MODULE ... WELCOME

static  calculation ....boring                 
number of steps                       =           10
timestep                              =  0.00000E+00  ps
time range                            =  0.00000E+00  ps
temperature                           =      1.00000   K
pressure                              =      0.00000 GPa
print thermo  periodicity             =            1

paralelisation - atoms decomposition
rank =         0   atoms       1 to    4000 load :     4000
=============================================================
properties at t=0

  Thermodynamic information 
  ---------------------------------------------
  step                  =         0
  time                  =  0.00000000E+00
  Ekin                  =  0.00000000E+00
  Temp                  =  0.00000000E+00
  Utot                  = -0.34102600E+05
  U_vdw                 = -0.34102600E+05
  U_coul                =  0.00000000E+00
  Pressure              =  0.35716516E+04
  Pvir_vdw              =  0.22292498E+02
  Pvir_coul             =  0.00000000E+00
  Pvir_tot              =  0.22292498E+02
  volume                =  0.36624913E+04
  a cell                =  0.15414361E+02
  b cell                =  0.15414361E+02
  c cell                =  0.15414361E+02
  ---------------------------------------------
  Etot                  = -0.34102600E+05
  Htot                  = -0.34102600E+05

stress tensor of initial configuration

TAU_NONB
  0.44584996E+02  0.34094247E-14  0.15735806E-14
  0.34094247E-14  0.44584996E+02  0.64934485E-14
  0.15735806E-14  0.64934485E-14  0.44584996E+02
iso =   0.44584996E+02(  0.11146249E-01)


TAU_TOTA
  0.44584996E+02  0.34094247E-14  0.15735806E-14
  0.34094247E-14  0.44584996E+02  0.64934485E-14
  0.15735806E-14  0.64934485E-14  0.44584996E+02
iso =   0.44584996E+02(  0.11146249E-01)


starting main loop
=============================================================
=============================================================
end of the main loop


stress tensor of final configuration

TAU_NONB
  0.44584996E+02  0.34094247E-14  0.15735806E-14
  0.34094247E-14  0.44584996E+02  0.64934485E-14
  0.15735806E-14  0.64934485E-14  0.44584996E+02
iso =   0.44584996E+02(  0.11146249E-01)


TAU_TOTA
  0.44584996E+02  0.34094247E-14  0.15735806E-14
  0.34094247E-14  0.44584996E+02  0.64934485E-14
  0.15735806E-14  0.64934485E-14  0.44584996E+02
iso =   0.44584996E+02(  0.11146249E-01)

=============================================================

TIME MODULE ... WELCOME

-------------------------------------------------------------
                           TOTAL :  cpu time     0.66
                          kardir :  cpu time     0.00
                          dirkar :  cpu time     0.00

-------------------------------------------------------------
main subroutines:
-------------------------------------------------------------
MD:
-------------------------------------------------------------
             engforce_nmlj       :  cpu time     0.35
-------------------------------------------------------------
=============================================================
\end{verbatim}

\subsection{Miniumim setting : Output files}
For this minimum setting, the code generates 4 files:

\begin{itemize}
\item OSZIFF : thermodynamic parameters 
\item TRAJFF : trajectory file (empty in this case) 
\item CONTFF : final configuration
\end{itemize}
theses files and other output files are described in a following section.

\newpage

\subsection{Input file : Description}

%###################################################################################################"
%                                         CONTROLTAG
%###################################################################################################"
\subsubsection{\&controltag }

Where : \verb?SUBROUTINE control_init? in \verb?control.f90?
\newline

\begin{longtable}{l|c|m{8cm}|m{2cm}}
\hline
\hline
Parameter        &  Type              &          Short Description                                                          & Default value \\
\hline
\hline
\rule[-2.5cm]{0cm}{5cm}
\verb?calc?      & (CHARACTER)        & \newline \verb?'md'?  : molecular dynamics simulation (see section~\ref{sec:MD}) \newline

                                        \verb?'opt'? : unconstrained optimization of configuation saved in TRAJFF file 
					(see section~\ref{sec:OPT}) \newline

					\verb?'vib'? : 'vib+fvib', 'vib+gmod', 'vib+band', 'vib+dos' : phonons related 
					(see section~\ref{sec:VIB})\newline

					\verb?'efg'? : Electric Field Gradient calculation from configuration in TRAJFF 
					(see section~\ref{sec:EFG})\newline

					\verb?'efg+acf?' : EFG auto-correlation function from EFGALL file 
					(see section~\ref{sec:EFG})\newline

					\verb?'gr'? : radial distribution function of configurations in TRAJFF
					(see section~\ref{sec:GR}) \newline                                                 & \verb?'md'?   \tabularnewline
\hline
\rule[-0.75cm]{0cm}{1.5cm}
\verb?lnmlj?     & (LOGICAL)          &  switch on the binary mixture lennard-jones potential (see section~\ref{sec:BMLJ})  & \verb?.FALSE.?  \\
\hline
\rule[-0.75cm]{0cm}{1.5cm}
\verb?lcoulomb?  & (LOGICAL)          &  switch on the coulombic potentials                                                 & \verb?.FALSE.? \\
\hline
\rule[-0.75cm]{0cm}{1.5cm}
\verb?lmorse?    & (LOGICAL)          &  switch on morse potential (to be tested )                                          & \verb?.FALSE.? \\
\hline
\rule[-0.75cm]{0cm}{1.5cm}
\verb?lbmhft?    & (LOGICAL)          &  switch on Born–Mayer–Huggins–Fumi–Tosi (BMHFT) potential                           & \verb?.FALSE.? \\
\hline
\rule[-0.75cm]{0cm}{1.5cm}
\verb?lbmhftd?   & (LOGICAL)          &  switch on Born–Mayer–Huggins–Fumi–Tosi  with damping functions                     & \verb?.FALSE.? \\
\hline
\rule[-0.75cm]{0cm}{1.5cm}
\verb?lvnlist?   & (LOGICAL)          &  use verlet neighbour list (see section~\ref{sec:vnlist})                           & \verb?.TRUE.?  \\
\hline
\rule[-0.75cm]{0cm}{1.5cm}
\verb?skindiff?  
                 & (REAL)             &  verlet-list cutoff ( cutoff + skindiff )                                           & \verb?0.1?\\
\hline
\rule[-0.75cm]{0cm}{1.5cm}
\verb?lstatic?   & (LOGICAL)          &  static calculation                                                                 & \verb?.FALSE.? \\
\hline
\rule[-0.75cm]{0cm}{1.5cm}
\verb?lcsvr?   & (LOGICAL)            &  switch on the stochastic velocity rescaling method by Bussi-Donadio-Parrinello     & \verb?.FALSE.? \\
\hline
\rule[-0.75cm]{0cm}{1.5cm}
\verb?lmsd?      & (LOGICAL)          & calculate mean square displacment on the fly during the md (see \verb?&msdtag? )    & \verb?.FALSE? \\
\hline
\rule[-0.75cm]{0cm}{1.5cm}
\verb?lvacf?     & (LOGICAL)          & calculate velocity autocorrelation function on the fly during the md 
                                        (see \verb?&vacftag? not the final version )                                        & \verb?.FALSE? \\
\hline
\rule[-0.75cm]{0cm}{1.5cm}
\verb?lsurf?     & (LOGICAL)          &  add "surface" contribution to electrostatic quantities (see \ref{sec:COUL})        & \verb?.FALSE.?  \\
\hline
\rule[-0.75cm]{0cm}{1.5cm}
\verb?lreduced?  & (LOGICAL)          &  \newline reduced units and reduced the output quantities by the 
                                         total number of particules \newline

					 Units are discussed in section \ref{sec:units} \newline                            & \verb?.FALSE.? \\ 
\hline
\rule[-0.75cm]{0cm}{1.5cm}
\verb?ltest?     & (LOGICAL)          &  get around the main code (for test purpose only)                                   & \verb?.FALSE.? \\
\hline
\rule[-0.75cm]{0cm}{1.5cm}
\verb?lrestart?   & (LOGICAL)         &  restart calculation and consider previous positions and velocities                 & \verb?.FALSE.? \\
\hline
\rule[-0.75cm]{0cm}{1.5cm}
\verb?restart_data?  
                 & (CHARACTER)        &  format of  \verb?POSFF? file : 'rvf' , 'rnn' , 'rnf' , 'rvn' \newline

                                         \verb?'rnn'? : positions only                   (ion\_type + 3 columns) \newline 
                                         \verb?'rnf'? : positions and forces             (ion\_type + 6 columns) \newline 
                                         \verb?'rvn'? : positions and velocities         (ion\_type + 6 columns) \newline 
                                         \verb?'rvf'? : positions, velocities and forces (ion\_type + 9 columns) 

                                                                                                                            & \verb?'rnn'? \tabularnewline
\hline
\rule[-0.75cm]{0cm}{1.5cm}
\verb?dgauss?    & (CHARACTER)        & \newline algorithm for gaussian distribution (no fondamental differences test purpose only) 
                                        (see section~\ref{sec:misc}) \newline  

                                        \verb?'boxmuller_basic'? : also known as the cartesian form  \newline

					\verb?'boxmuller_polar'? : supposed to be faster than the basic method because it is 
					simpler to compute.     \newline

					\verb?'knuth'? : knuth algorithm cited by~\cite{B-ALLEN_TILDESLEY} \newline         & \verb?'boxmuller_basic'? \tabularnewline
\hline
\rule[-0.75cm]{0cm}{1.5cm}
\verb?longrange? & (CHARACTER)        & \newline algorithm for long-range interactions. It will be used for coulombic potential 
                                        calculation and Electric-Field Gradient calculation, if both are set together. \newline

					\verb?'direct'? : Direct summation (can be used for EFG but difficult to converge for columbic potential) \newline

					\verb?'ewald'? : Ewald summation technique \newline                                 &  \verb?'ewald?'      \tabularnewline
\hline
\rule[-0.75cm]{0cm}{1.5cm}
\verb?cutshortrange?    
                 & (REAL)             &  real-space cut-off for short-range interactions  
                                         [(lnmlj, lmorse, lbmhft, lmhftd )  & \verb?-?\\
\hline
\rule[-0.75cm]{0cm}{1.5cm}
\verb?cutlongrange?     
                 & (REAL)             &  real-space cut-off for long-range interactions ( lcoulomb )                        & \verb?-?\\
\hline
\rule[-0.75cm]{0cm}{1.5cm}
\verb?ltraj?     & (LOGICAL)          &  save trajectory (pos, vel, forces) during MD                                       & \verb?.FALSE.? \\
\hline
\rule[-0.75cm]{0cm}{1.5cm}
\verb?itraj_start?   
                 & (INTEGER)          &  write trajectory from step \verb?itraj_start?                                      & \verb?1? \\
\hline
\rule[-0.75cm]{0cm}{1.5cm}
\verb?itraj_period?  
                 & (INTEGER)          &  write trajectory each \verb?itraj_period? steps                                    & \verb?10000? \\
\hline
\rule[-0.75cm]{0cm}{1.5cm}
\verb?itraj_format?  
                 & (INTEGER)          &  format of TRAJFF file ( = 0 BINARY, = 1 FORMATED )                                 & \verb?1? \\
\hline
\rule[-0.75cm]{0cm}{1.5cm}
\verb?traj_data? & (CHARACTER)        &  type of data in TRAJFF (see \verb?restart_data?)                                   & \verb?'rnn'? \\
\hline
\rule[-0.75cm]{0cm}{1.5cm}
\verb?iscff_data?& (CHARACTER)        &  type of data in ISCFF (see \verb?restart_data?)                                    & \verb?'rnn'? \\
\hline
\rule[-0.75cm]{0cm}{1.5cm}
\verb?iscff_format? 
                 & (INTEGER)          &  format of TRAJFF file ( = 0 BINARY, = 1 FORMATED )                                 & \verb?1? \\
\hline
\rule[-0.75cm]{0cm}{1.5cm}
\verb?iefall_format? 
                 & (INTEGER)          &  format of EFALL file ( = 0 BINARY, = 1 FORMATED )                                  & \verb?1? \\
\hline
\rule[-0.75cm]{0cm}{1.5cm}
\verb?iefgall_format? 
                 & (INTEGER)          &  format of EFGALL file ( = 0 BINARY, = 1 FORMATED )                                 & \verb?1? \\
\hline
\rule[-0.75cm]{0cm}{1.5cm}
\verb?idipall_format? 
                 & (INTEGER)          &  format of DIPFF file ( = 0 BINARY, = 1 FORMATED )                                  & \verb?1? \\
\hline
\hline
\end{longtable}

%###################################################################################################"
%                                         MDTAG
%###################################################################################################"
\subsubsection{\&mdtag}

Where : \verb?SUBROUTINE md_init? in \verb?md.f90?
\newline

\begin{longtable}{l|c|m{8cm}|m{2cm}}
\hline
\hline
Parameter        &  Type              &          Short Description                                                          & Default value \\
\hline
\hline
\rule[-0.75cm]{0cm}{1.5cm}
\verb?integrator?& (CHARACTER)        &  \newline algorithm for dynamic integration (see section~\ref{sec:integrator}) \newline                   

                                         \verb?'nve-vv'?    : NVE velocity-verlet \newline
					  
					 \verb?'nve-lf'?    : NVE leap-frog \newline
					   
				         \verb?'nve-be'?    : NVE beeman \newline

					 \verb?'nve-lfq'?   : NVE leap-frog +  equilibration \newline                        

					 \verb?'npe-vv'?    : NPE beredsen barostat \newline                        
					 
					 \verb?'nvt-and'?   : NVT Andersen \newline
					 
					 \verb?'nvt-nhc2'?  : NVT Nosé-Hoover two chains \newline

					 \verb?'nvt-nhcn'?  : NVT Nosé-Hoover chains \newline

					 \verb?'npt-nhcnp'? : NPT Martyna-Tuckerman-Tobias-Klein + Nosé-Hoover thermostat \newline 
                                                                                                                            & \verb?'nve-vv'? \tabularnewline
					 
\hline
\rule[-0.75cm]{0cm}{1.5cm}
\verb?setvel?    & (CHARACTER)        &  \newline velocity distribution \newline 
                                        
					 \verb?'MaxwBoltz'? : Maxwell-Boltzmann distribution \newline 

					 \verb?'uniform'?   : Uniform distribution (test purpose) \newline                  & \verb?'MaxwBoltz'? \tabularnewline
\hline
\rule[-0.75cm]{0cm}{1.5cm}
\verb?npas?      & (INTEGER)          &  total number of time steps                                                         & \verb?10?    \\
\hline
\rule[-0.75cm]{0cm}{1.5cm}
\verb?nequil?    & (INTEGER)          &  total number of equilibration steps                                                & \verb?10? \\
\hline
\rule[-0.75cm]{0cm}{1.5cm}
\verb?nequil_period? 
                 & (INTEGER)          &  equilibration period                                                               & \verb?1? \\
\hline
\rule[-0.75cm]{0cm}{1.5cm}
\verb?annealing? & (REAL)             &  rescaling factor for annealing velocities                                          & \verb?1.0? \\
\hline
\rule[-0.75cm]{0cm}{1.5cm}
\verb?nprint?    & (INTEGER)          &  print thermo info to standard output (period)                                      & \verb?1? \\
\hline
\rule[-0.75cm]{0cm}{1.5cm}
\verb?fprint?    & (INTEGER)          &  print thermo info to file OSZIFF (period)                                          & \verb?1? \\
\hline
\rule[-0.75cm]{0cm}{1.5cm}
\verb?npropr?    & (INTEGER)          &  period to calculate on-the-fly properties (\verb?lmsd,lvacf?)                      & \verb?1? \\
\hline
\rule[-0.75cm]{0cm}{1.5cm}
\verb?npropr_start?   
                 & (INTEGER)          &  starting point to calculate on-the-fly properties (\verb?lmsd,lvacf?)              & \verb?0? \\
\hline
\rule[-0.75cm]{0cm}{1.5cm}
\verb?spas?      & (INTEGER)          &  save configuration each \verb?spas? steps in file CONTFF                           & \verb?1000? \\
\hline
\rule[-0.75cm]{0cm}{1.5cm}
\verb?dt?        & (REAL)             &  time step for integration [ps]                                                     & \verb?0.001? \\
\hline
\rule[-0.75cm]{0cm}{1.5cm}
\verb?temp?      & (REAL)             &  temperature [K]                                                                    & \verb?1.0?\\
\hline
\rule[-0.75cm]{0cm}{1.5cm}
\verb?press?     & (REAL)             &  pressure [GPa]                                                                     & \verb?0.0?\\
\hline
\rule[-0.75cm]{0cm}{1.5cm}
\verb?timesca_thermo? 
                 & (REAL)             &  time scale of thermostat [ps]                                                      & \verb?NULL? \\
\hline
\rule[-0.75cm]{0cm}{1.5cm}
\verb?timesca_baro?   
                 & (REAL)             &   time scale of barostat [ps]                                                       & \verb?NULL? \\
\hline
\rule[-0.75cm]{0cm}{1.5cm}
\verb?taucsvr?   & (REAL)             &  characteristic time in Stochastic velocity rescale method by 
                                         Bussi-Donadio-Parrinello (see \verb?&controltag lcsvr?)                            & \verb?-? \\
\hline
\rule[-0.75cm]{0cm}{1.5cm}
\verb?nuandersen?     
                 & (REAL)             &  characteristic frequency in andersen thermostat (obsolete)                         & \verb?NULL? \\
\hline
\rule[-0.75cm]{0cm}{1.5cm}
\verb?tauTberendsen?  
                 & (REAL)             &  characteristic time in berendsen thermostat [ps] 
                                         (simple rescale if tauberendsen = dt )                                             & \verb?dt? \\
\hline
\rule[-0.75cm]{0cm}{1.5cm}
\verb?tauPberendsen?  
                 & (REAL)             &  characteristic time in berendsen barostat [ps]
                                         (simple rescale if tauberendsen = dt )                                             & \verb?dt? \\
\hline
\rule[-0.75cm]{0cm}{1.5cm}
\verb?nhc_n?     & (INTEGER)          &  Nose-Hoover Chain : number of chains (extended variable)                           & \verb?4? \\
\hline
\rule[-0.75cm]{0cm}{1.5cm}
\verb?nhc_yosh_order? 
                 & (INTEGER)          &  Nose-Hoover Chain : order of the yoshida integrator 
                                         $(2n+1)$ \, from $n=~$0 to 4                                  & \verb?3? \\
\hline
\rule[-0.75cm]{0cm}{1.5cm}
\verb?nhc_mults?      
                 & (INTEGER)          &  Nose-Hoover Chain : number of multiple timesteps                                   & \verb?2? \\
\hline
\hline
\end{longtable}

%###################################################################################################"
%                                         FIELDTAG
%###################################################################################################"

\subsubsection{\&fieldtag}

Where : \verb?SUBROUTINE field_init? in \verb?field.f90?
\newline

\begin{longtable}{l|c|m{8cm}|m{2cm}}
\hline
\hline
Parameter        &  Type              &          Short Description                                                          & Default value \\
\hline
\hline
\rule[-0.75cm]{0cm}{1.5cm}
\verb?lKA?       &  (LOGICAL)         & use the Kob-Andersen model for BMLJ (see \S \ref{sec:KA})                           & \verb?.FALSE.? \\
\hline
\rule[-0.75cm]{0cm}{1.5cm}
\verb?ctrunc?    &  (CHARACTER)       & \newline truncation and shift of the potential \newline 

                                                    \verb?'notrunc'? : no shift of the truncation \newline 
                                                    \verb?'linear'?  : linear correction (potential 
                                                                       discontinuity at $r=r_{cut}$) \newline 
                   			            \verb?'quadratic'?: quadratic correction ( potential and force 
                                                    discontinuity at $r=r_{cut}$) \newline                                  & \verb?'notrunc'? \tabularnewline
\hline
\rule[-0.75cm]{0cm}{1.5cm}
\verb?symmetric_pot?         
                 & (LOGICAL)          &  make all potentials symmetric ( default .true. but who knows ?) \newline  
	                                   only the upper triangle in matrices in needed                                    & \verb?.TRUE.? \tabularnewline
\hline
\rule[-0.75cm]{0cm}{1.5cm}
\verb?mass(X)?   &  (REAL)            & mass of the particule type with X = 1 , 2 \ldots \verb?ntype? [a.m.u]               & \verb?1.0? \\
\hline
\rule[-0.75cm]{0cm}{1.5cm}
\verb?quad_efg(X)? 
                 &  (REAL)            & quadrupolar moment of nucleus for NMR interaction for type $X$ with $X=~$1,2 \ldots \verb?ntype? [mb]     
                                                                                                                            & \verb?1.0? \\
\hline
\rule[-0.75cm]{0cm}{1.5cm}
                 &                    & if \verb?lcoulomb=.TRUE.? & \\
\hline
\rule[-0.75cm]{0cm}{1.5cm}
\verb?doefield?         
                 & (LOGICAL)          &  calculate electric field (internaly switch on for self-consistent calculation  $\ge$ dipole) 
                                                                                                                            & \verb?.FALSE.? \\
\hline
\rule[-0.75cm]{0cm}{1.5cm}
\verb?doefg?         
                 & (LOGICAL)          &  calculate electric field gradient (internaly switch on for self-consistent calculation $\ge$ quadrupole)     
                                                                                                                            & \verb?.FALSE.? \\
\hline
\rule[-0.75cm]{0cm}{1.5cm}
\verb?qch(X)?    &  (REAL)            & charge of the particule type with X = 1 , 2 \ldots \verb?ntype? [e]                 & \verb?0.0?\\
\hline
\rule[-0.75cm]{0cm}{1.5cm}
\verb?dip(X,3)?  &  (REAL)            & static dipole on particule of type $X$ with $X=$~1 , 2 \ldots \verb?ntype? [e\AA]   & \verb?0.0?\\
\hline
\rule[-0.75cm]{0cm}{1.5cm}
\verb?quad(X,3,3)?
                 &  (REAL)            & static quadrupole on particule of type $X$ with $X=$~1 , 2 \ldots \verb?ntype? [e\AA$^2$]  
	                                                                                                                    & \verb?0.0?\\
\hline
\rule[-0.75cm]{0cm}{1.5cm}
\verb?ncelldirect?           
                 &  (INTEGER)         & \newline number of neighboring cells in the direct summation 
                                         (total number of cells $=ncelldirect^3$ \newline 

                                          for \verb?lcoulomb=.TRUE.? and \verb?longrange='direct'?  \newline                  
					  
					  not efficient for coulombic quantities $<$ electric field gradient order          & \verb?2? \\
\hline
\rule[-0.75cm]{0cm}{1.5cm}
\verb?kES(3)?    &  (INTEGER)         & \newline number of kpoints in the reciprocal contribution of the Ewald summation
                                         (total number of cells $=ncellewald^3$ \newline                                      
					 
					 for \verb?lcoulomb=.TRUE.? and \verb?longrange='ewald'?  \newline                  & \verb?10 10 10? \\
\hline
\rule[-0.75cm]{0cm}{1.5cm}
\verb?alphaES?   &  (REAL)            & Ewald screening parameter                                                           & \verb?1.0? \\
\hline
\rule[-0.75cm]{0cm}{1.5cm}
\verb?lautoES?   & (LOGICAL)          & auto-determination of Ewald parameter (\verb?kES?, \verb?alphaES?) 
                                        from \verb?epsw? (accuracy) and \verb?cutlongrange? ( accuracy )                    & \verb?.FALSE.? \\
\hline
\rule[-0.75cm]{0cm}{1.5cm}
\verb?epsw?      &  (REAL)            & targeted error in the ewald summation (see \ref{sec:COUL} $\epsilon_\omega$) 
                                                                                                                            & \verb?1.0? \\
\hline
\hline
\rule[-0.75cm]{0cm}{1.5cm}
                 &                    & if \verb?lnmlj=.TRUE.?                                                              & \\
\hline
\rule[-0.75cm]{0cm}{1.5cm}
\verb?epslj(X,Y)?& (REAL)             & depth of the potential well of the Lennard jones potential 
                                        with X,Y = 1 , 2 \ldots \verb?ntype? [eV]                                           & \verb?1.0? \tabularnewline
\hline
\rule[-0.75cm]{0cm}{1.5cm}
\verb?sigmalj(X,Y)?          
                 & (REAL)             & finite distance at which the inter-particle potential is zero with 
                                          X,Y = 1 , 2 \ldots \verb?ntype? [\AA]                                             & \verb?1.0? \\
\hline
\rule[-0.75cm]{0cm}{1.5cm}
\verb?qlj(X,Y)?  & (REAL)             & exponent of the repulsive part with X,Y = 1 , 2 \ldots \verb?ntype?                 & \verb?12? \\
\hline
\rule[-0.75cm]{0cm}{1.5cm}
\verb?plj(X,Y)?  & (REAL)             & exponent of the attractive part with X,Y = 1 , 2 \ldots \verb?ntype?                & \verb?6? \\
\hline
\hline
\rule[-0.75cm]{0cm}{1.5cm}
                 &                    & if \verb?lbmhft=.TRUE.? or \verb?lbmhftd=.TRUE.?                                    & \\
\hline
\rule[-0.75cm]{0cm}{1.5cm}
\verb?Abmhftd(X,Y)?          
                 & (REAL)             & parameter of the Born-Meyer-Huggins potential 
                                        with X,Y = 1 , 2 \ldots \verb?ntype? [eV]                                           & \verb?0.0? \\
\hline
\rule[-0.75cm]{0cm}{1.5cm}
\verb?Bbmhftd(X,Y)?          
                 & (REAL)             & parameter of the Born-Meyer-Huggins potential 
                                        with X,Y = 1 , 2 \ldots \verb?ntype? [\AA$^1$]                                      & \verb?0.0? \\
\hline
\rule[-0.75cm]{0cm}{1.5cm}
\verb?Cbmhftd(X,Y)?          
                 & (REAL)             & parameter of the Born-Meyer-Huggins potential
                                        with X,Y = 1 , 2 \ldots \verb?ntype? [eV\AA$^6$]                                    & \verb?0.0? \\
\hline
\rule[-0.75cm]{0cm}{1.5cm}
\verb?Dbmhftd(X,Y)?          
                 & (REAL)             & parameter of the Born-Meyer-Huggins potential
                                        with X,Y = 1 , 2 \ldots \verb?ntype? [eV\AA$^8$]                                    & \verb?0.0? \\
\hline
\rule[-0.75cm]{0cm}{1.5cm}
\verb?BDbmhftd(X,Y)?         
                 & (REAL)             &  damping terms of dispersion contribution only \verb?lbmhftd=.TRUE.?                & \verb?0.0? \\
\hline
\hline
\rule[-0.75cm]{0cm}{1.5cm}
                 &                    & for polarizabilities                                                                &  \\

\hline
\rule[-0.75cm]{0cm}{1.5cm}
\verb?algo_pol?  & (CHARACTER)        & \verb?'scf'? : self-consistent determination of induced field from polarizabilities
                                        \verb?'cg'? :  dev. stage                                                           & \verb?.FALSE.?\\
\hline
\rule[-0.75cm]{0cm}{1.5cm}
\verb?ldip_polar(X)? 
                 & (LOGICAL)          & switch (dipole) polarizabilities for ion X= 1 , 2 \ldots \verb?ntype?               & \verb?.FALSE.?\\
\hline
\rule[-0.75cm]{0cm}{1.5cm}
\verb?lquad_polar(X)? 
                 & (LOGICAL)          & switch (quadrupole) polarizabilities for ion X= 1 , 2 \ldots \verb?ntype?           & \verb?.FALSE.?\\
\hline
\rule[-0.75cm]{0cm}{1.5cm}
\verb?poldip(X,3,3)?
                 & (REAL)             & dipole-dipole polarizability tensor for type X=1 , 2 \ldots \verb?ntype?            & \verb?0.0? \\
\hline
\rule[-0.75cm]{0cm}{1.5cm}
\verb?polquad(X,3,3,3)?
                 & (REAL)             & quadrupole-quadrupole polarizability tensor for type X=1 , 2 \ldots \verb?ntype?    & \verb?0.0? \\
\hline
\rule[-0.75cm]{0cm}{1.5cm}
\verb?poldip_iso(X)?
                 & (REAL)             & dipole-dipole polarizability isotropic part only for type X=1 , 2 \ldots \verb?ntype?            
                                                                                                                            & \verb?0.0? \\
\hline
\rule[-0.75cm]{0cm}{1.5cm}
\verb?polquad_iso(X)?
                 & (REAL)             & quadrupole-quadrupole polarizability tensor for type X=1 , 2 \ldots \verb?ntype?    & \verb?0.0? \\
\hline
\rule[-0.75cm]{0cm}{1.5cm}
\verb?ldip_damp(X,3,3)?      
                 & (LOGICAL)          & switch on damping function on dipole for type X=1 , 2 \ldots \verb?ntype? (see \ref{sec:pola})         
                                                                                                                            & \verb?.FALSE.? \\
\hline
\rule[-0.75cm]{0cm}{1.5cm}
\verb?dip_pol_damp_b(X,3,3)?     
                 & (REAL)             & $b$ parameter of damping function on dipole for type X=1 , 2 \ldots \verb?ntype? (see \ref{sec:pola})
                                                                                                                            & \verb?0.0? \\
\hline
\rule[-0.75cm]{0cm}{1.5cm}
\verb?dip_pol_damp_c(X,3,3)?     
                 & (REAL)             & $c$ parameter of damping function on dipole for type X=1 , 2 \ldots \verb?ntype? (see \ref{sec:pola})
                                                                                                                            & \verb?0.0? \\
\hline
\rule[-0.75cm]{0cm}{1.5cm}
\verb?dip_pol_damp_k(X,3,3)?     
                 & (INTEGER)          & $k$ parameter of damping function on dipole for type X=1 , 2 \ldots \verb?ntype? (see \ref{sec:pola})
                                                                                                                            & \verb?4? \\
\hline
\hline
\end{longtable}

%###################################################################################################"
%                                         OPTTAG
%###################################################################################################"
\subsubsection{\&opttag}

Where : \verb?SUBROUTINE opt_init? in \verb?opt.f90?
\newline

\begin{longtable}{l|c|m{8cm}|m{2cm}}
\hline
\hline
Parameter        &  Type              &          Short Description                                                          & Default value \\
\hline
\hline
\rule[-0.75cm]{0cm}{1.5cm}
\verb?optalgo?   & (CHARACTER)        & \newline choose unconstrained optimization algorithm \newline 

                                        \verb?'sastry'? : Linear minimisation via cubic extrapolation of potential 
					Original author S. Sastry JNCASR \newline

					\verb?'lbfgs'? : Limited memory BFGS method for large scale optimisation
					Author: J. Nocedal~\addref  \newline 
					 
					\verb?'m1qn3'? : M1QN3, Version 3.3, October 2009
					Authors: Jean Charles Gilbert, Claude Lemarechal, INRIA.~\addref \newline            & \verb?sastry? \tabularnewline
\hline
\rule[-0.75cm]{0cm}{1.5cm}
\verb?ncopt?     & (INTEGER)          & number of configurations in TRAJFF                                                   & \verb?0? \\
\hline
\rule[-0.75cm]{0cm}{1.5cm}
\verb?nskipopt?  & (INTEGER)          & number of configurations skipped in the beginning of TRAJFF                          & \verb?0? \\
\hline
\rule[-0.75cm]{0cm}{1.5cm}
\verb?nmaxopt?   & (INTEGER)          & number of configurations to be optimize                                              & \verb?1? \\
\hline
\hline
\end{longtable}

%###################################################################################################"
%                                         VIBTAG
%###################################################################################################"
\subsubsection{\&vibtag}

Where : \verb?SUBROUTINE vib_init? in \verb?vib.f90?
\newline

\begin{longtable}{l|c|m{8cm}|m{2cm}}
\hline
\hline
Parameter        &  Type              &          Short Description                                                          & Default value \\
\hline
\hline
\rule[-0.75cm]{0cm}{1.5cm}
\verb?lwrite_vectff? 
                 & (LOGICAL)          & write the vector field (i.e eigenvectors) in VECTFF file                            & \verb?.FALSE.?  \\
\hline
\rule[-0.75cm]{0cm}{1.5cm}
\verb?ncvib?     & (INTEGER)          & number of configurations in ISCFF to be analysed                                    & \verb?0? \\
\hline
\rule[-0.75cm]{0cm}{1.5cm}
\verb?ngconf?    & (INTEGER)          & number of configurations generated with \verb?calc='vib+gmod'?                      & \verb?0? \\
\hline
\rule[-0.75cm]{0cm}{1.5cm}
\verb?imod?      & (INTEGER)          & if \verb?calc='vib+gmod'? generate the mode \verb?imod?                             & \verb?4? \\
\hline
\rule[-0.75cm]{0cm}{1.5cm}
\verb?nkphon?    & (INTEGER)          & number of kpoint between ks and kf used when calc = 'vib+band'                      & \verb?NULL? \\
\hline
\rule[-0.75cm]{0cm}{1.5cm}
\verb?resdos?    & (REAL)             & resolution in vibrational density of states distribution [\AA]                      & \verb?1.0? \\
\hline
\rule[-0.75cm]{0cm}{1.5cm}
\verb?omegamax?  & (REAL)             & maximum value in dos                                                                & \verb?100.0? \\
\hline
\rule[-0.75cm]{0cm}{1.5cm}
\verb?ksx,ksy,ksz?  
                 & (REAL)             & first kpoint                                                                        & \verb?NULL? \\
\hline
\rule[-0.75cm]{0cm}{1.5cm}
\verb?kfx,kfy,kfz?  
                 & (REAL)             & last kpoint                                                                         & \verb?NULL? \\
\hline
\hline
\end{longtable}

%###################################################################################################"
%                                         EFGTAG
%###################################################################################################"

\subsubsection{\&efgtag}

Where : \verb?SUBROUTINE efg_init? in \verb?efg.f90?
\newline

\begin{longtable}{l|c|m{8cm}|m{2cm}}
\hline
\hline
Parameter        &  Type              &          Short Description                                                          & Default value \\
\hline
\hline
\rule[-0.75cm]{0cm}{1.5cm}
\verb?lefgprintall?    
                 & (LOGICAL)          &  whether or not we want to print all the efg for each atoms and 
                                               configurations in file EFGALL                                                & \verb?.FALSE.? \\
\hline
\rule[-0.75cm]{0cm}{1.5cm}
\verb?lefg_it_contrib? 
                 & (LOGICAL)          & if one wants to get the contribution to EFG separated in types 
                                              (only \verb?'direct'? and test purpose)                                       & \verb?.FALSE.? \\
\hline
\rule[-0.75cm]{0cm}{1.5cm}
\verb?ncefg?     & (INTEGER)          & number of configurations read in TRAJFF for EFG calculation ( \verb?calc='efg'? )   & \verb?0? \\
\hline
\rule[-0.75cm]{0cm}{1.5cm}
\verb?ntcor?     & (INTEGER)          & maximum number of steps for the efg auto-correlation function( \verb?calc='efg+acf'?)
                                                                                                                            & \verb?10? \\
\hline
\rule[-0.75cm]{0cm}{1.5cm}
\verb?resvzz?    & (REAL)             & resolution in $V_{zz}$ distribution                                                 & \verb?0.1? \\
\hline
\rule[-0.75cm]{0cm}{1.5cm}
\verb?reseta?    & (REAL)             & resolution in $\eta_Q$ distribution                                                 & \verb?0.1? \\
\hline
\rule[-0.75cm]{0cm}{1.5cm}
\verb?resu?      & (REAL)             & resolution in $U_i$ ditribution                                                     & \verb?0.1? \\
\hline
\rule[-0.75cm]{0cm}{1.5cm}
\verb?vzzmin?    & (REAL)             & minimum value of $V_{zz}$ distribution which is then between [vzzmin, -vzzmin]      & \verb?-4.0? \\ 
\hline
\rule[-0.75cm]{0cm}{1.5cm}
\verb?umin?      & (REAL)             & minimum value of umin distribution which is then between [umin, -umin]              & \verb?-4.0? \\
\hline
\hline
\end{longtable}

%###################################################################################################"
%                                         MSDTAG
%###################################################################################################"

\subsubsection{\&msdtag}

Where : \verb?SUBROUTINE msd_init? in \verb?msd.f90?
BE CAREFUL !!! NOT HEAVELY TESTED !
\newline

\begin{longtable}{l|c|m{8cm}|m{2cm}}
\hline
\hline
Parameter        &  Type              &          Short Description                                                          & Default value \\
\hline
\hline
\rule[-0.75cm]{0cm}{1.5cm}
\verb?nblock?    & (INTEGER)          & ???????? I forgot ;)                                                                & \verb?10? \\
\hline
\rule[-0.75cm]{0cm}{1.5cm}
\verb?ibmax?     &  (INTEGER)         & ???????? I forgot ;)                                                                & \verb?20? \\
\hline
\rule[-0.75cm]{0cm}{1.5cm}
\verb?tdifmax?   & (REAL)             & ???????? I forgot ;)                                                                & \verb?100.0? \\
\hline
\hline
\end{longtable}

%###################################################################################################"
%                                         GRTAG
%###################################################################################################"

\subsubsection{\&grtag}

Where : \verb?SUBROUTINE control_init? in \verb?radial_distribution.f90?
\newline

\begin{longtable}{l|c|m{8cm}|m{2cm}}
\hline
\hline
Parameter        &  Type              &          Short Description                                                          & Default value \\
\hline
\hline
\rule[-0.75cm]{0cm}{1.5cm}
\verb?nc?        & (INTEGER)          & number of configurations to read in TRAJFF                                          & \verb?0? \\
\rule[-0.75cm]{0cm}{1.5cm}
\verb?nskip?     & (INTEGER)          & number of configurations skipped in the beginning of TRAJFF                         & \verb?0? \\
\rule[-0.75cm]{0cm}{1.5cm}
\verb?resg?      & (REAL)             & resolution in the radial distribution function                                      & \verb?0.1? \\
\hline
\hline
\end{longtable}

%###################################################################################################"
%                                         VACFTAG
%###################################################################################################"
\subsubsection{\&vacftag}

Where : \verb?SUBROUTINE vacf_init? in \verb?vacf.f90?
BE CAREFUL
\newline

\begin{longtable}{l|c|m{8cm}|m{2cm}}
\hline
\hline
Parameter        &  Type              &          Short Description                                                          & Default value \\
\hline
\hline
\rule[-0.75cm]{0cm}{1.5cm}
\verb?it0?       & (INTEGER)          & ???????? I forgot ;)                                                                & \verb?1? \\
\hline
\rule[-0.75cm]{0cm}{1.5cm}
\verb?tdifmax?   & (REAL)             & ???????? I forgot ;)                                                                & \verb?100.0? \\ 
\hline
\hline
\end{longtable}

\subsubsection{\&stochiotag }
\subsubsection{\&vois1tag }
\subsubsection{\&rmctag }

%###################################################################################################"
%                                         OUTFILE DESCRIPTION
%###################################################################################################"
\section{Output files description}

\bibliography{mdff}
\bibliographystyle{amsplain}

\end{document}
