%MDFF documentation
\documentclass[a4paper]{article}
\usepackage{lscape}
\usepackage{geometry}
\usepackage{chngpage}
\usepackage[utf8]{inputenc}
\usepackage[T1]{fontenc}
%\usepackage[frenchb]{babel}
\usepackage{multirow}
\usepackage{array}
\usepackage{longtable}
\usepackage{color}


%##############################################
%new commands
%##############################################
\newcommand{\MDFF}{{\sc MDff}}
\newcommand{\FMV}{{\sc fmv}}
\newcommand{\wiki}{\textcolor{red}{[wiki]}} 
\newcommand{\addref}{\textcolor{blue}{[ref]}}

\newenvironment{changemargin}[2]{\begin{list}{}{%
\renewcommand{\arraystretch}{4.0}
\setlength{\topsep}{0pt}%
\setlength{\leftmargin}{0pt}%
\setlength{\rightmargin}{0pt}%
\setlength{\listparindent}{\parindent}%
\setlength{\itemindent}{\parindent}%
\setlength{\parsep}{0pt plus 1pt}%
\setlength{\tabcolsep}{1cm}
\setlength{\extrarowheight}{40pt}
\addtolength{\leftmargin}{#1}%
\addtolength{\rightmargin}{#2}%
}\item }{\end{list}}


\title{Molecular Dynamics FF, Manual , version 0.2.2}

\begin{document}

\maketitle
\clearpage
\tableofcontents

\clearpage

\section{Aim}

Le but de ce document est de repertorier les differents 
algorithmes utilis\'es par \MDFF. Notamment, definir les 
commandes (TAG) et leurs valeurs par défaut. Mais également, 
les r\'eferences aux codes, articles originaux, livres dont 
sont extraits les diff\'erents algorithmes, ou preciser dans 
le cas contraire (i.e \FMV!) ce que fait la routine point par point. 
Ce travail de documentation permettra de faciliter la d\'ecouverte
de bug qui independamment de ma volont\'e sont (et seront) 
forc\'ement pr\'esents. 

\clearpage

\section{Theory}

\subsection{Molecular Dynamics\label{sec:MD}}

\subsection{Optimisation\label{sec:OPT}}

\subsection{Phonons\label{sec:VIB}}

\subsection{Electric-Field Gradient\label{sec:EFG}}

\subsection{Radial distribution function\label{sec:GR}}

\subsection{Binary Mixture Lennar-Jones\label{sec:BMLJ}}

\subsection{Kob-Andersen model\label{sec:KA}}

\subsection{Verlet list\label{sec:vnlist}}

\subsection{Units \label{sec:units}}


\section{Usage}

\MDFF only needs one input file usually called control.F~\footnote{the .F extension permits to get nice colors with vim} 

\subsection{Minimum settings: the input file}

The input file is composed of different sections. 
Each section is defined by  \&SECTIONNAME and \&END.
Everything outside the sections is ignored.
\\
\\
In \MDFF, five sections are always required :

\begin{itemize}
\item \&controltag ... \&end $\rightarrow$ General control parameters for calculation 
\item \&mdtag ... \&end      $\rightarrow$ General Molecular dynamics parameters.
\item \&configtag ... \&end  $\rightarrow$ Information on the configuration
\item \&fieldtag ... \&end   $\rightarrow$ Set the Force-Field.
\item \&proptag ... \&end    $\rightarrow$ Calculation of properties
\end{itemize}


Using this minimum settings (5 empty sections) the code runs with default values.
It correspond to a molecular dynamics trajectory of 10 unit of times of a fcc lennard-jones 
crystal (256 atoms) at $\rho=$1.0 at $T=$1.0. In the next sextion we show the standard output of this minimum setting. 

\clearpage
\subsection{Minimum setting: standart output}
\begin{verbatim}
                            \\|//                    
                           -(o o)-                           
========================oOO==(_)==OOo========================
          ____    ____  ______   ________  ________  
         |_   \  /   _||_   _ `.|_   __  ||_   __  | 
           |   \/   |    | | `. \ | |_ \_|  | |_ \_| 
           | |\  /| |    | |  | | |  _|     |  _|    
          _| |_\/_| |_  _| |_.' /_| |_     _| |_     
         |_____||_____||______.'|_____|   |_____|    
  
=============================================================

MOLECULAR DYNAMICS ...for fun                 
mdff.0.2.0
parallel version
filipe.manuel.vasconcelos@gmail.com  
Running on   1 nodes                  
time     : 2011/11/05    15:50:14
calc     : md                                                          

=============================================================

periodic boundary conditions in cubic cell     
verlet list used 
units reduced by the number of atom
dynamic calculation
NVE ensemble --- velocity verlet integrator    
with equilibration:             
berendsen scaling ( is not canonical ...and so NVE)
number of steps                      =         10
timestep                             = .10000E-02
time range                           = .10000E-01
temperature                          =    1.00000
number of equilibration steps        =         10
equilibration period                 =          1
Berendsen thermo time scale          = .10000E-02
tauberendsen = dt -> simple rescale
print thermo  periodicity            =          1

=============================================================

structure generated from the code
Face-centered cubic structure :  4x  4x  4  
system                               : fcc                                                         
natm                                 =        256
cell parameter                       =     6.3496
volume                               =   256.0000
density                              =     1.0000
distance check subroutine
smallest distance =   1.12

=============================================================

force field information :                      
no masses are implemented                      
LENNARD-JONES                  

       eps    /    / sigma* \ q       / sigma*  \ p \
 V = ------- |  p | ------- |    - q | -------- |   |
      q - p   \    \   r    /         \    r    /   /

USER DEFINED MODEL                             
cutoff  =    2.50000

A-A interactions:
sigmaAA =    1.00000
epsAA   =    1.00000
qAA     =   12.00000
pAA     =    6.00000
long range correction :         -0.535433102
 
paralelisation - atom decomposition
rank    0 atom    1 to  256
=============================================================
... MD output
=============================================================

                 TOTAL :  cpu time     0.15
                    MD :  cpu time     0.01

main subroutines:
MD:
    engforce_bmlj      :  cpu time     0.01
    engforce_coul_DS   :  cpu time     0.00
    vnlistcheck        :  cpu time     0.00
=============================================================
\end{verbatim}

\subsection{Miniumim setting : Output files}
For this minimum setting, the code generates 4 files:

\begin{itemize}
\item OUTFF  : close to the standard output
\item OSZIFF : thermodynamic parameters 
\item TRAJFF : trajectory file (empty in this case) 
\item CONTFF : final configuration
\end{itemize}
theses files and other output files are described in a following section.

\subsection{Input file : Description}

%###################################################################################################"
%                                         CONTROLTAG
%###################################################################################################"
\subsubsection{\&controltag }

Where : \verb?SUBROUTINE control_init? in \verb?control.f90?
\newline

\begin{longtable}{l|c|m{8cm}|m{2cm}}
\hline
\hline
Parameter        &  Type              &          Short Description                                                          & Default value \\
\hline
\hline
\rule[-0.75cm]{0cm}{1.5cm}
\verb?lbmlj?     & (LOGICAL)          &  set the binary mixture lennard-jones potential (see section~\ref{sec:BMLJ})        & \verb?.TRUE.?  \\
\hline
\rule[-0.75cm]{0cm}{1.5cm}
\verb?lcoulomb?  & (LOGICAL)          &  set coulombic potentials (to be tested)                                            & \verb?.FALSE.? \\
\hline
\rule[-0.75cm]{0cm}{1.5cm}
\verb?lvnlist?   & (LOGICAL)          &  use verlet neighbour list (see section~\ref{sec:vnlist})                           & \verb?.TRUE.?  \\
\hline
\rule[-0.75cm]{0cm}{1.5cm}
\verb?lstatic?   & (LOGICAL)          &  static calculation                                                                 & \verb?.FALSE.? \\
\hline
\rule[-0.75cm]{0cm}{1.5cm}
\verb?lpbc?      & (LOGICAL)          &  use periodic boundary conditions (if .FALSE. should be tested)                     & \verb?.TRUE.?  \\
\hline
\rule[-0.75cm]{0cm}{1.5cm}
\verb?lreduced?  & (LOGICAL)          &  \newline reduced the output quantities by the total number of particules \newline

                                         Comment: this has nothing to do with the units of the calculated quantities
					 Units are discussed in section \ref{sec:units} \newline                            & \verb?.TRUE.? \\ 
\hline
\rule[-0.75cm]{0cm}{1.5cm}
\verb?lshiftpot? & (LOGICAL)          &  use a shifted potential (maybe redundant with ctrunc)                              & \verb?.TRUE.? \\
\hline
\rule[-0.75cm]{0cm}{1.5cm}
\verb?ltest?     & (LOGICAL)          &  get round of the main code (for test purpose only)                                 & \verb?.FALSE.? \\
\hline
\rule[-0.75cm]{0cm}{1.5cm}
\verb?lrestart?   & (LOGICAL)         &  restart calculation read previous positions and velocities                         & \verb?.FALSE.? \\
\hline
\rule[-2.5cm]{0cm}{5cm}
\verb?calc?      & (CHARACTER)        & \newline \verb?'md'?  : molecular dynamics simulation (see section~\ref{sec:MD}) \newline

                                        \verb?'opt'? : unconstrained optimization of configuation saved in TRAJFF file 
					(see section~\ref{sec:OPT}) \newline

					\verb?'vib'? : 'vib+fvib', 'vib+gmod', 'vib+band', 'vib+dos' : phonons related 
					(see section~\ref{sec:VIB})\newline

					\verb?'efg'? : Electric Field Gradient calculation from configuration in TRAJFF 
					(see section~\ref{sec:EFG})\newline

					\verb?'efg+acf?' : EFG auto-correlation function from EFGALL file 
					(see section~\ref{sec:EFG})\newline

					\verb?'gr'? : radial distribution function of configurations in TRAJFF
					(see section~\ref{sec:GR}) \newline    & \verb?'md'?   \tabularnewline
\hline
\rule[-0.75cm]{0cm}{1.5cm}
\verb?dgauss?    & (CHARACTER)        & \newline algorithm for gaussian distribution (no fondamental differences test purpose only) 
                                        (see section~\ref{sec:rand}) \newline  

                                        \verb?'boxmuller_basic'? : also known as the cartesian form  \newline

					\verb?'boxmuller_polar'? : supposed to be faster than the basic method because it is 
					simpler to compute.     \newline

					\verb?'knuth'? : knuth algorithm cited by~\cite{B-ALLEN_TILDESLEY} \newline          & \verb?'boxmuller_basic'? \tabularnewline
\hline
\rule[-0.75cm]{0cm}{1.5cm}
\verb?longrange? & (CHARACTER)        & \newline algorithm for long-range interactions. It will be used for coulombic potential 
                                        calculation and Electric-Field Gradient calculation, if both are set together. \newline

					\verb?'direct'? : Direct summation (can be used for EFG but difficult to converge for columbic potential) \newline

					\verb?'ewald'? : Ewald summation technique \newline                                &  \verb?'ewald?'      \tabularnewline
\hline
\rule[-0.75cm]{0cm}{1.5cm}
\verb?cutoff?    & (DBLE)             &  real-space cut-off for lennard-jones potentials                                   & \verb?2.5?\\
\hline
\rule[-0.75cm]{0cm}{1.5cm}
\verb?skindiff?  & (DBLE)             &  verlet-list cutoff ( cutoff + skindiff )                                          & \verb?0.1?\\
\hline
\hline
\end{longtable}

%###################################################################################################"
%                                         MDTAG
%###################################################################################################"
\subsubsection{\&mdtag}

Where : \verb?SUBROUTINE md_init? in \verb?md.f90?
\newline

\begin{longtable}{l|c|m{8cm}|m{2cm}}
\hline
\hline
Parameter        &  Type              &          Short Description                                                          & Default value \\
\hline
\hline
\rule[-0.75cm]{0cm}{1.5cm}
\verb?ltraj?         & (LOGICAL)      &  save trajectory in file TRAJFF                                                     & \verb?.FALSE.? \\
\hline
\rule[-0.75cm]{0cm}{1.5cm}
\verb?integrator?    & (CHARACTER)    &  \newline algorithm for dynamic integration (see section~\ref{sec:integrator}) \newline                   

                                         \verb?'nve-vv'? : NVE velocity-verlet \newline
					  
					 \verb?'nve-lf'? : NVE leap-frog \newline
					   
				         \verb?'nve-be'? : NVE beeman \newline
					 
					 \verb?'nvt-and'?: NVT Andersen \newline
					 
					 \verb?'nvt-nh'? : NVT Nosé-Hoover \newline
					 
					 \verb?'nvt-nhc2'? : NVT Nosé-Hoover two chains \newline
					 
					 \verb?'nve-lfq'? : NVE leap-frog +  equilibration \newline                         & \verb?'nve-vv'? \tabularnewline
\hline
\rule[-0.75cm]{0cm}{1.5cm}
\verb?setvel?        & (CHARACTER)    &  \newline velocity distribution \newline 
                                        
					 \verb?'MaxwBoltz'? : Maxwell-Boltzmann distribution \newline 

					 \verb?'uniform'?   : Uniform distribution (test purpose) \newline                  & \verb?'MaxwBoltz'? \tabularnewline
\hline
\rule[-0.75cm]{0cm}{1.5cm}
\verb?npas?          & (INTEGER)      &  total number of time steps                                                         & \verb?10?    \\
\hline
\rule[-0.75cm]{0cm}{1.5cm}
\verb?nequil?        & (INTEGER)      &  total number of equilibration steps                                                & \verb?10? \\
\hline
\rule[-0.75cm]{0cm}{1.5cm}
\verb?nequil_period? & (INTEGER)      &  equilibration period                                                               & \verb?1? \\
\hline
\rule[-0.75cm]{0cm}{1.5cm}
\verb?nprint?        & (INTEGER)      &  print thermo info to standard output and OUTFF (period)                            & \verb?1? \\
\hline
\rule[-0.75cm]{0cm}{1.5cm}
\verb?fprint?        & (INTEGER)      &  print thermo info to file OSZIFF (period)                                          & \verb?1? \\
\hline
\rule[-0.75cm]{0cm}{1.5cm}
\verb?itraj_start?   & (INTEGER)      &  write trajectory from step \verb?itraj_start?                                      & \verb?1? \\
\hline
\rule[-0.75cm]{0cm}{1.5cm}
\verb?itraj_period?  & (INTEGER)      &  write trajectory each \verb?itraj_period? steps                                    & \verb?10000? \\
\hline
\rule[-0.75cm]{0cm}{1.5cm}
\verb?spas?          & (INTEGER)      &  save configuration each \verb?spas? steps in file CONTFF                           & \verb?1000? \\
\hline
\rule[-0.75cm]{0cm}{1.5cm}
\verb?dt?            & (DBLE)         &  time step for integration                                                          & \verb?0.001? \\
\hline
\rule[-0.75cm]{0cm}{1.5cm}
\verb?temp?          & (DBLE)         &  temperature                                                                        & \verb?1.0?\\
\hline
\rule[-0.75cm]{0cm}{1.5cm}
\verb?nuandersen?    & (DBLE)         &  characteristic frequency in andersen thermostat                                    & \verb?NULL? \\
\hline
\rule[-0.75cm]{0cm}{1.5cm}
\verb?tauberendsen?  & (DBLE)         &  characteristic time in berendsen thermostat (simple rescale if tauberendsen = dt ) & \verb?dt? \\
\hline
\rule[-0.75cm]{0cm}{1.5cm}
\verb?Qnosehoover?   & (DBLE)         &  Q parameter in Nose-Hoover two chains                                              & \verb?NULL? \\
\hline
\hline
\end{longtable}

%###################################################################################################"
%                                         CONFIGTAG
%###################################################################################################"
\subsubsection{\&configtag}

Where : \verb?SUBROUTINE config_init? in \verb?config.f90?
\newline

\begin{longtable}{l|c|m{8cm}|m{2cm}}
\hline
\hline
Parameter        &  Type              &          Short Description                                                          & Default value \\
\hline
\hline
\rule[-0.75cm]{0cm}{1.5cm}
\verb?lgenconf?  & (LOGICAL)          & \newline if \verb?.TRUE.? generate the configuration \newline

                                      if \verb?.FALSE.? read configuration from file \newline

				      POSFF if \verb?calc='md'? \newline

				      TRAJFF if \verb?calc='opt'?or \verb?'efg'? \newline

				      ISCFF if \verb?calc='vib'?  \newline                                                  & \verb?.TRUE.? \tabularnewline
\hline
\rule[-0.75cm]{0cm}{1.5cm}
\verb?lfcc?      & (LOGICAL)         & for \verb?lgenconf=.TRUE.?  generate face-centered cubic lattice                     & \verb?.TRUE.? \\
\hline
\rule[-0.75cm]{0cm}{1.5cm}
\verb?lsc?       & (LOGICAL)         & for \verb?lgenconf=.TRUE.?  generate simple cubic lattice (primitive)                & \verb?.FALSE.? \\
\hline
\rule[-0.75cm]{0cm}{1.5cm}
\verb?lbcc?      & (LOGICAL)         & for \verb?lgenconf=.TRUE.?  generate body-centered cubic lattice                     & \verb?.FALSE.? \\
\hline
\rule[-0.75cm]{0cm}{1.5cm}
\verb?system?    & (CHARACTER)       & system name                                                                          & \verb?'UNKNOWN'? \\
\hline
\rule[-0.75cm]{0cm}{1.5cm}
\verb?struct?    & (CHARACTER)       & \newline crystal structure for fcc (only make sense for \verb?ntype=2?) \newline
           
	                              \verb?'NaCl'? \newline
				      
				      \verb?'random'? \newline                                                              & \verb?'random'? \tabularnewline
\hline
\rule[-0.75cm]{0cm}{1.5cm}
\verb?ntype?     & (INTEGER)         & number of types \verb?=1? or \verb?=2?                                               & \verb?1? \\
\hline
\rule[-0.75cm]{0cm}{1.5cm}
\verb?natm?      & (INTEGER)         & total number of atoms                                                                & \verb?NULL? \\
\hline
\rule[-0.75cm]{0cm}{1.5cm}
\verb?ncell?     & (INTEGER)         & number of cell in 1D (with \verb?lfcc?,\verb?lsc?,\verb?lbcc?)                       & \verb?4? \\
\hline
\rule[-0.75cm]{0cm}{1.5cm}
\verb?xna?       & (DBLE)            & relative composition of atom of type A                                               & \verb?1.0? \\
\hline
\rule[-0.75cm]{0cm}{1.5cm}
\verb?xnb?       & (DBLE)            & relative composition of atom of type B                                               & \verb?0.0? \\
\hline
\rule[-0.75cm]{0cm}{1.5cm}
\verb?rho?       & (DBLE)            & density number ( if \verb?box=0.0? )                                                 & \verb?NULL? \\
\hline
\rule[-0.75cm]{0cm}{1.5cm}
\verb?box?       & (DBLE)            & lattice parameter ( if \verb?rho=0.0? )                                              & \verb?NULL? \\
\hline
\hline
\hline
\end{longtable}

%###################################################################################################"
%                                         FIELDTAG
%###################################################################################################"

\subsubsection{\&fieldtag}

Where : \verb?SUBROUTINE field_init? in \verb?field.f90?
\newline

\begin{longtable}{l|c|m{8cm}|m{2cm}}
\hline
\hline
Parameter          &  Type              &          Short Description                                                          & Default value \\
\hline
\hline
\rule[-0.75cm]{0cm}{1.5cm}
\verb?lKA?         &  (LOGICAL)         & use the Kob-Andersen model for BMLJ\ref{sec:KA}                                     & \verb?.FALSE.? \\
\hline
\rule[-0.75cm]{0cm}{1.5cm}
\verb?ctrunc?      &  (CHARACTER)       & \newline truncation and shift of the potential \newline 

                                         \verb?'linear'?: linear correction (potential discontinuity at $r=r_{cut}$) \newline 
					 
                   			 \verb?'quadratic'?: quadratic correction ( potential and force 
					 discontinuity at $r=r_{cut}$) \newline                                               & \verb?'linear'? \tabularnewline
\hline
\rule[-0.75cm]{0cm}{1.5cm}
\verb?ncelldirect? &  (INTEGER)         & \newline number of neighboring cells in the direct summation 
                                         (total number of cells $=ncelldirect^3$ \newline 

                                          for \verb?lcoulomb=.TRUE.? and \verb?longrange='direct'?  \newline                  
					  
					  not efficient for coulombic interaction                                             & \verb?2? \\
\hline
\rule[-0.75cm]{0cm}{1.5cm}
\verb?ncellewald?  &  (INTEGER)         & \newline number of kpoints in the reciprocal contribution of the Ewald summation
                                         (total number of cells $=ncellewald^3$ \newline                                      
					 
					 for \verb?lcoulomb=.TRUE.? and \verb?longrange='ewald'?  \newline                   & \verb?10? \\
\hline
\rule[-0.75cm]{0cm}{1.5cm}
\verb?alphaES?     &  (DBLE)            & Ewald screening parameter                                                          & \verb?1.0? \\
\hline
\rule[-0.75cm]{0cm}{1.5cm}
\verb?epsXY?       &  (DBLE)            & depth of the potential well of the Lennard jones potential. XY = AA, BB or AB      & \verb?1.0? \\
\hline
\rule[-0.75cm]{0cm}{1.5cm}
\verb?sigmaXY?     &  (DBLE)            & finite distance at which the inter-particle potential is zero. XY = AA, BB or AB   & \verb?1.0? \\
\hline
\rule[-0.75cm]{0cm}{1.5cm}
\verb?qljXY?       &  (DBLE)            & exponent of the repulsive part. XY = AA, BB or AB                                  & \verb?12? \\
\hline
\rule[-0.75cm]{0cm}{1.5cm}
\verb?pljXY?       &  (DBLE)            & exponent of the attratcive part. XY = AA, BB or AB                                 & \verb?6? \\
\hline
\rule[-0.75cm]{0cm}{1.5cm}
\verb?mX?          &  (DBLE)            & mass of the particule. X= A or B (not used)                                        & \verb?1.0? \\
\hline
\rule[-0.75cm]{0cm}{1.5cm}
\verb?qX?          &  (DBLE)            & charge of the particule. X= A or B                                                 & \verb?qA = 1.0?\newline\verb?qB =-1.0?\\
\hline
\hline
\end{longtable}

%###################################################################################################"
%                                         PROPTAG
%###################################################################################################"
\subsubsection{\&proptag}

Where : \verb?SUBROUTINE prop_init? in \verb?prop.f90?
\newline

\begin{longtable}{l|c|m{8cm}|m{2cm}}
\hline
\hline
Parameter          &  Type              &          Short Description                                                          & Default value \\
\hline
\hline
\rule[-0.75cm]{0cm}{1.5cm}
\verb?lefg?        & (LOGICAL)          & calculate electric field gradient on the fly during the md                          & \verb?.FALSE.? \\
\hline
\rule[-0.75cm]{0cm}{1.5cm}
\verb?lgr?         & (LOGICAL)          & calculate radial distribution function on the fly during the md                     & \verb?.FALSE? \\
\hline
\rule[-0.75cm]{0cm}{1.5cm}
\verb?lstrfac?     & (LOGICAL)          & calculate static structural factor on the fly during the md                         & \verb?.FALSE? \\
\hline
\rule[-0.75cm]{0cm}{1.5cm}
\verb?lmsd?        & (LOGICAL)          & calculate mean square displacment on the fly during the md                          & \verb?.FALSE? \\
\hline
\rule[-0.75cm]{0cm}{1.5cm}
\verb?lvacf?       & (LOGICAL)          & calculate velocity autocorrelation function on the fly during the md                & \verb?.FALSE? \\
\hline
\rule[-0.75cm]{0cm}{1.5cm}
\verb?nprop_start? & (INTEGER)          & calculate properties from step \verb?nprop_start?                                   & \verb?0? \\
\hline
\rule[-0.75cm]{0cm}{1.5cm}
\verb?nprop?       & (INTEGER)          & calculate properties each \verb?nprop? step ( for lgr and efg not sure anymore)     & \verb?1? \\
\hline
\rule[-0.75cm]{0cm}{1.5cm}
\verb?nprop_print? & (INTEGER)          & print properties each \verb?nprop_print?                                            & \verb?nprop? \\
\hline
\hline
\end{longtable}

%###################################################################################################"
%                                         OPTTAG
%###################################################################################################"
\subsubsection{\&opttag}

Where : \verb?SUBROUTINE opt_init? in \verb?opt.f90?
\newline

\begin{longtable}{l|c|m{8cm}|m{2cm}}
\hline
\hline
Parameter        &  Type              &          Short Description                                                          & Default value \\
\hline
\hline
\rule[-0.75cm]{0cm}{1.5cm}
\verb?optalgo?   & (CHARACTER)        & \newline choose unconstrained optimization algorithm \newline 

                                        \verb?'sastry'? : Linear minimisation via cubic extrapolation of potential 
					Original author S. Sastry JNCASR \newline

					\verb?'lbfgs'? : Limited memory BFGS method for large scale optimisation
					Author: J. Nocedal~\addref  \newline 
					 
					\verb?'m1qn3'? : M1QN3, Version 3.3, October 2009
					Authors: Jean Charles Gilbert, Claude Lemarechal, INRIA.~\addref \newline            & \verb?sastry? \tabularnewline
\hline
\rule[-0.75cm]{0cm}{1.5cm}
\verb?ncopt?     & (INTEGER)          & number of configurations in TRAJFF                                                   & \verb?0? \\
\hline
\rule[-0.75cm]{0cm}{1.5cm}
\verb?nskipopt?  & (INTEGER)          & number of configurations skipped in the beginning                                    & \verb?0? \\
\hline
\rule[-0.75cm]{0cm}{1.5cm}
\verb?nmaxopt?   & (INTEGER)          & number of configurations to be optimize                                              & \verb?1? \\
\hline
\hline
\end{longtable}

%###################################################################################################"
%                                         VIBTAG
%###################################################################################################"
\subsubsection{\&vibtag}

Where : \verb?SUBROUTINE vib_init? in \verb?vib.f90?
\newline

\begin{longtable}{l|c|m{8cm}|m{2cm}}
\hline
\hline
Parameter            &  Type              &          Short Description                                                          & Default value \\
\hline
\hline
\rule[-0.75cm]{0cm}{1.5cm}
\verb?lwrite_vectff? & (LOGICAL)          & write the vector field (i.e eigenvectors) in VECTFF file                            & \verb?.FALSE.?  \\
\hline
\rule[-0.75cm]{0cm}{1.5cm}
\verb?ncvib?         & (INTEGER)          & number of configurations in ISCFF to be analysed                                    & \verb?0? \\
\hline
\rule[-0.75cm]{0cm}{1.5cm}
\verb?ngconf?        & (INTEGER)          & number of configurations generated with \verb?calc='vib+gmod'?                      & \verb?0? \\
\hline
\rule[-0.75cm]{0cm}{1.5cm}
\verb?imod?          & (INTEGER)          & if \verb?calc='vib+gmod'? generate the mode \verb?imod?                             & \verb?4? \\
\hline
\rule[-0.75cm]{0cm}{1.5cm}
\verb?nkphon?        & (INTEGER)          & number of kpoint between ks and kf used when calc = 'vib+band'                      & \verb?NULL? \\
\hline
\rule[-0.75cm]{0cm}{1.5cm}
\verb?resdos?       & (DBLE)              & resolution in vibrational density of states distribution                            & \verb?1.0? \\
\hline
\rule[-0.75cm]{0cm}{1.5cm}
\verb?omegamax?     & (DBLE)              & maximum value in dos                                                                & \verb?100.0? \\
\hline
\rule[-0.75cm]{0cm}{1.5cm}
\verb?ksx,ksy,ksz?  & (DBLE)              & first kpoint & \verb?NULL? \\
\hline
\rule[-0.75cm]{0cm}{1.5cm}
\verb?kfx,kfy,kfz?  & (DBLE)              & last kpoint  & \verb?NULL? \\
\hline
\hline
\end{longtable}

%###################################################################################################"
%                                         EFGTAG
%###################################################################################################"

\subsubsection{\&efgtag}

Where : \verb?SUBROUTINE efg_init? in \verb?efg.f90?
\newline

\begin{longtable}{l|c|m{8cm}|m{2cm}}
\hline
\hline
Parameter              &  Type              &          Short Description                                                          & Default value \\
\hline
\hline
\rule[-0.75cm]{0cm}{1.5cm}
\verb?lefgprintall?    & (LOGICAL)          &  whether or not we want to print all the efg for each atoms and 
                                               configurations in file EFGALL                                                      & \verb?.FALSE.? \\
\hline
\rule[-0.75cm]{0cm}{1.5cm}
\verb?lefg_it_contrib? & (LOGICAL)          & if one wants to get the contribution to EFG separated in types 
                                              (only \verb?'direct'? and test purpose)                                             & \verb?.FALSE.? \\
\hline
\rule[-0.75cm]{0cm}{1.5cm}
\verb?ncefg?           & (INTEGER)          & number of configurations read in TRAJFF for EFG calculation ( \verb?calc='efg'? )    & \verb?0? \\
\hline
\rule[-0.75cm]{0cm}{1.5cm}
\verb?ntcor?           & (INTEGER)          & maximum number of steps for the efg auto-correlation function( \verb?calc='efg+acf'?)& \verb?10? \\
\hline
\rule[-0.75cm]{0cm}{1.5cm}
\verb?ncellewald?      & (INTEGER)          & \newline number of kpoints in the reciprocal contribution of the Ewald summation
                                              (total number of cells $=ncellewald^3$ \newline
					 
                                              for \verb?longrange='ewald'?  \newline                                               & \verb?0? \\
\hline
\rule[-0.75cm]{0cm}{1.5cm}
\verb?ncelldirect?     & (INTEGER)          & \newline number of neighboring cells in the direct summation
                                              (total number of cells $=ncelldirect^3$ \newline
					 
                                              for \verb?longrange='direct'?  only \newline                                         & \verb?0? \\
\hline
\rule[-0.75cm]{0cm}{1.5cm}
\verb?cutefg?          & (DBLE)             & cut-off distance for efg calculation in for direct summation 
                                              (\verb?longrange='direct'?) & \verb?30.0? \\
\hline
\rule[-0.75cm]{0cm}{1.5cm}
\verb?alphaES?         & (DBLE)             & Ewald sum screening parameter                                                        & \verb?1.0? \\
\hline
\rule[-0.75cm]{0cm}{1.5cm}
\verb?resvzz?          & (DBLE)             & resolution in vzz distribution                                                       & \verb?0.1? \\
\hline
\rule[-0.75cm]{0cm}{1.5cm}
\verb?reseta?          & (DBLE)             & resolution in eta distribution                                                       & \verb?0.1? \\
\hline
\rule[-0.75cm]{0cm}{1.5cm}
\verb?resu?            & (DBLE)             & resolution in Ui ditribution                                                         & \verb?0.1? \\
\hline
\rule[-0.75cm]{0cm}{1.5cm}
\verb?vzzmin?          & (DBLE)             & minimum value of vzz distribution which is then between [vzzmin, -vzzmin]            & \verb?-4.0? \\ 
\hline
\rule[-0.75cm]{0cm}{1.5cm}
\verb?umin?            & (DBLE)             & minimum value of umin distribution which is then between [umin, -umin]               & \verb?-4.0? \\
\hline
\hline
\end{longtable}

%###################################################################################################"
%                                         MSDTAG
%###################################################################################################"

\subsubsection{\&msdtag}

Where : \verb?SUBROUTINE msd_init? in \verb?msd.f90?
BE CAREFUL !!!
\newline

\begin{longtable}{l|c|m{8cm}|m{2cm}}
\hline
\hline
Parameter        &  Type              &          Short Description                                                          & Default value \\
\hline
\hline
\rule[-0.75cm]{0cm}{1.5cm}
\verb?nblock?    & (INTEGER)          & ???????? I forgot ;)                                                                & \verb?10? \\
\hline
\rule[-0.75cm]{0cm}{1.5cm}
\verb?ibmax?     &  (INTEGER)         & ???????? I forgot ;)                                                                & \verb?20? \\
\hline
\rule[-0.75cm]{0cm}{1.5cm}
\verb?tdifmax?   & (DBLE)             & ???????? I forgot ;)                                                                & \verb?100.0? \\
\hline
\hline
\end{longtable}

%###################################################################################################"
%                                         GRTAG
%###################################################################################################"

\subsubsection{\&grtag}

Where : \verb?SUBROUTINE control_init? in \verb?radial_distribution.f90?
\newline

\begin{longtable}{l|c|m{8cm}|m{2cm}}
\hline
\hline
Parameter        &  Type              &          Short Description                                                          & Default value \\
\hline
\hline
\rule[-0.75cm]{0cm}{1.5cm}
\verb?resg?      & (DBLE)             & resolution in the radial distribution function                                      & \verb?? \\
\hline
\hline
\end{longtable}

%###################################################################################################"
%                                         VACFTAG
%###################################################################################################"
\subsubsection{\&vacftag}

Where : \verb?SUBROUTINE vacf_init? in \verb?vacf.f90?
BE CAREFUL
\newline

\begin{longtable}{l|c|m{8cm}|m{2cm}}
\hline
\hline
Parameter        &  Type              &          Short Description                                                          & Default value \\
\hline
\hline
\rule[-0.75cm]{0cm}{1.5cm}
\verb?it0?       & (INTEGER)          & ???????? I forgot ;)                                                                & \verb?1? \\
\hline
\rule[-0.75cm]{0cm}{1.5cm}
\verb?tdifmax?   & (DBLE)             & ???????? I forgot ;)                                                                & \verb?100.0? \\ 
\hline
\hline
\end{longtable}


%###################################################################################################"
%                                         OUTFILE DESCRIPTION
%###################################################################################################"
\section{Output files description}

\bibliography{mdff}
\bibliographystyle{amsplain}

\end{document}
