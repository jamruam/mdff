%MDFF documentation
\documentclass[10pt,a4paper]{article}
\usepackage{lscape}
\usepackage{geometry}
\usepackage{chngpage}
%\usepackage{pslatex}
\usepackage[utf8]{inputenc}
\usepackage[T1]{fontenc}
\usepackage[frenchb]{babel}
%\usepackage{graphicx}
%\usepackage{bm}
%\providecommand{\bm}[1]{\mathbf{#1}}
%\usepackage{amsmath}
%\usepackage{color}

%##############################################
%new commands
%##############################################
\newcommand{\MDFF}{{\sc MDff}}
\newcommand{\FMV}{{\sc fmv}}
\newcommand{\wiki}{\textcolor{red}{[wiki]}} 
\newcommand{\addref}{\textcolor{blue}{[ref]}}

\newenvironment{changemargin}[2]{\begin{list}{}{%
\setlength{\topsep}{0pt}%
\setlength{\leftmargin}{0pt}%
\setlength{\rightmargin}{0pt}%
\setlength{\listparindent}{\parindent}%
\setlength{\itemindent}{\parindent}%
\setlength{\parsep}{0pt plus 1pt}%
\addtolength{\leftmargin}{#1}%
\addtolength{\rightmargin}{#2}%
}\item }{\end{list}}


\title{Molecular Dynamics FF, Manual , version 0.2.2}

\begin{document}

\maketitle

\clearpage

\section{But}

Le but de ce document est de repertorier les differents 
algorithmes utilis\'es par \MDFF. Notamment, definir les 
commandes (TAG) et leurs valeurs par défaut. Mais également, 
les r\'eferences aux codes, articles originaux, livres dont 
sont extraits les diff\'erents algorithmes, ou preciser dans 
le cas contraire (i.e \FMV!) ce que fait la routine point par point. 
Ce travail de documentation permettra de faciliter la d\'ecouverte
de bug qui independamment de ma volont\'e sont (et seront) 
forc\'ement pr\'esents. 

\clearpage


\section{Usage}

\MDFF only needs one input file usually called control.F~\footnote{the .F extension permits to get nice colors with vim} 

\subsection{Minimum settings: the input file}

The input file is composed of different sections. 
Each section is defined by  \&SECTIONNAME and \&END.
Everything outside the sections is ignored.
\\
\\
In \MDFF, five sections are always required :

\begin{itemize}
\item \&controltag ... \&end $\rightarrow$ General control parameters for calculation 
\item \&mdtag ... \&end      $\rightarrow$ General Molecular dynamics parameters.
\item \&configtag ... \&end  $\rightarrow$ Information on the configuration
\item \&fieldtag ... \&end   $\rightarrow$ Set the Force-Field.
\item \&proptag ... \&end    $\rightarrow$ Calculation of properties
\end{itemize}


Using this minimum settings (5 empty sections) the code runs with default values.
It correspond to a molecular dynamics trajectory of 10 unit of times of a fcc lennard-jones 
crystal (256 atoms) at $\rho=$1.0 at $T=$1.0. In the next sextion we show the standard output of this minimum setting. 

\clearpage
\subsection{Minimum setting: standart output}
\begin{verbatim}
                            \\|//                    
                           -(o o)-                           
========================oOO==(_)==OOo========================
          ____    ____  ______   ________  ________  
         |_   \  /   _||_   _ `.|_   __  ||_   __  | 
           |   \/   |    | | `. \ | |_ \_|  | |_ \_| 
           | |\  /| |    | |  | | |  _|     |  _|    
          _| |_\/_| |_  _| |_.' /_| |_     _| |_     
         |_____||_____||______.'|_____|   |_____|    
  
=============================================================

MOLECULAR DYNAMICS ...for fun                 
mdff.0.2.0
parallel version
filipe.manuel.vasconcelos@gmail.com  
Running on   1 nodes                  
time     : 2011/11/05    15:50:14
calc     : md                                                          

=============================================================

periodic boundary conditions in cubic cell     
verlet list used 
units reduced by the number of atom
dynamic calculation
NVE ensemble --- velocity verlet integrator    
with equilibration:             
berendsen scaling ( is not canonical ...and so NVE)
number of steps                      =         10
timestep                             = .10000E-02
time range                           = .10000E-01
temperature                          =    1.00000
number of equilibration steps        =         10
equilibration period                 =          1
Berendsen thermo time scale          = .10000E-02
tauberendsen = dt -> simple rescale
print thermo  periodicity            =          1

=============================================================

structure generated from the code
Face-centered cubic structure :  4x  4x  4  
system                               : fcc                                                         
natm                                 =        256
cell parameter                       =     6.3496
volume                               =   256.0000
density                              =     1.0000
distance check subroutine
smallest distance =   1.12

=============================================================

force field information :                      
no masses are implemented                      
LENNARD-JONES                  

       eps    /    / sigma* \ q       / sigma*  \ p \
 V = ------- |  p | ------- |    - q | -------- |   |
      q - p   \    \   r    /         \    r    /   /

USER DEFINED MODEL                             
cutoff  =    2.50000

A-A interactions:
sigmaAA =    1.00000
epsAA   =    1.00000
qAA     =   12.00000
pAA     =    6.00000
long range correction :         -0.535433102
 
paralelisation - atom decomposition
rank    0 atom    1 to  256
=============================================================
...
=============================================================

                 TOTAL :  cpu time     0.15
                    MD :  cpu time     0.01

main subroutines:
MD:
    engforce_bmlj      :  cpu time     0.01
    engforce_coul_DS   :  cpu time     0.00
    vnlistcheck        :  cpu time     0.00
=============================================================
\end{verbatim}

\subsection{Miniumim setting : Output files}
For this minimum setting, the code generates 4 files:

\begin{itemize}
\item OUTFF  : close to the standard output
\item OSZIFF : thermodynamic parameters 
\item TRAJFF : trajectory file (empty in this case) 
\item CONTFF : final configuration
\end{itemize}
theses files and other output files are described in a following section.

\subsection{Input file : Description}

%###################################################################################################"
%                                         CONTROLTAG
%###################################################################################################"
\subsubsection{\&controltag }

Where : \verb?SUBROUTINE control_init? in \verb?control.f90?
\newline

\begin{tabular}{lcc}
\\
\verb?lbmlj?     & (LOGICAL)          &   \\
\\
\verb?lcoulomb?  & (LOGICAL)          &   \\
\\
\verb?lvnlist?   & (LOGICAL)          &   \\
\\
\verb?lstatic?   & (LOGICAL)          &   \\
\\
\verb?calc?      & (CHARACTER)        &   \\
\\
\verb?dgauss?    & (CHARACTER)        &   \\
\\
\verb?lpbc?      & (LOGICAL)          &   \\
\\
\verb?longrange? & (CHARACTER)        &   \\
\\
\verb?cutoff?    & (DBLE) &   \\
\\
\verb?skindiff?  & (DBLE) &   \\
\\
\verb?restart?   & ()                 &   \\
\\
\verb?lreduced?  & (LOGICAL)          &   \\
\\
\verb?lshiftpot? & (LOGICAL)          &   \\
\\
\verb?ltest?     & (LOGICAL)          &   \\
\\
\end{tabular}

%###################################################################################################"
%                                         MDTAG
%###################################################################################################"
\subsubsection{\&mdtag}

Where : \verb?SUBROUTINE md_init? in \verb?md.f90?

\begin{tabular}{lcc}
\\
\verb?npas? & (INTEGER)  & \\
\\
\verb?integrator? & (CHARACTER)  & \\
\\
\verb?dt? & (DBLE)  & \\
\\
\verb?temp? & (DBLE)  & \\
\\
\verb?nequil? & (INTEGER)  & \\
\\
\verb?nequil_period? & (INTEGER) & \\
\\
\verb?nprint? & (INTEGER) & \\
\\
\verb?fprint? & (INTEGER) & \\
\\
\verb?ltraj? & (LOGICAL) & \\
\\
\verb?itraj_start? & (INTEGER) & \\
\\
\verb?itraj_period? & (INTEGER) & \\
\\
\verb?spas? & (INTEGER) & \\
\\
\verb?setvel? & (CHARACTER) & \\
\\
\verb?nuandersen? & (DBLE) & \\
\\
\verb?tauberendsen? & (DBLE) & \\
\\
\verb?Qnosehoover? & (DBLE) & \\
\\
\end{tabular}

%###################################################################################################"
%                                         CONFIGTAG
%###################################################################################################"
\subsubsection{\&configtag}

Where : \verb?SUBROUTINE config_init? in \verb?config.f90?

\begin{tabular}{lcc}
\\
\verb?system? & (CHARACTER) & \\
\\
\verb?rho? & (DBLE)  & \\
\\
\verb?box? & (DBLE)  & \\
\\
\verb?ntype? & (INTEGER)  & \\
\\
\verb?natm? & (INTEGER) & \\
\\
\verb?struct? & (CHARACTER) & \\
\\
\verb?lfcc? & (LOGICAL) & \\
\\
\verb?lsc? & (LOGICAL) & \\
\\
\verb?lbcc? & (LOGICAL) & \\
\\
\verb?ncell? & (INTEGER) & \\
\\
\verb?xna? & (DBLE) & \\
\\
\verb?xnb? & (DBLE) & \\
\\
\verb?lgenconf? & (LOGICAL) & \\
\\
\end{tabular}

%###################################################################################################"
%                                         FIELDTAG
%###################################################################################################"

\subsubsection{\&fieldtag}

Where : \verb?SUBROUTINE field_init? in \verb?field.f90?

\begin{tabular}{lcc}
\\
\verb?epsAA? &   (DBLE)  & \\
\\
\verb?epsAB? &   (DBLE)  & \\
\\
\verb?epsBB? &  (DBLE)  & \\
\\
\verb?sigmaAA? & (DBLE)  & \\
\\
\verb?sigmaAB? & (DBLE)  & \\
\\
\verb?sigmaBB? & (DBLE)  & \\
\\
\verb?qljAA? &  (DBLE)  & \\
\\
\verb?pljAA? &  (DBLE)  & \\
\\
\verb?qljAB? &  (DBLE)  & \\
\\
\verb?pljAB? &  (DBLE)  & \\
\\
\verb?qljBB? &  (DBLE)  & \\
\\
\verb?pljBB? &  (DBLE)  & \\
\\
\verb?mA?   &   (DBLE)  & \\
\\
\verb?mB?   &   (DBLE)  & \\
\\
\verb?qa?   &    (DBLE)  & \\
\\
\verb?qb?   &    (DBLE)  & \\
\\
\verb?ncelldirect? & (INTEGER)  & \\
\\
\verb?ncellewald?  &  (INTEGER)  & \\
\\
\verb?lKA?         &  (LOGICAL)  & \\
\\
\verb?alphaES?     &  (DBLE)  & \\
\\
\end{tabular}

%###################################################################################################"
%                                         PROPTAG
%###################################################################################################"
\subsubsection{\&proptag}

Where : \verb?SUBROUTINE prop_init? in \verb?prop.f90?

\begin{tabular}{lcc}
\\
\verb?nprop_start? & (INTEGER)  & \\
\\
\verb?nprop?       & (INTEGER)  & \\
\\
\verb?nprop_print? & (INTEGER)  & \\
\\
\verb?lefg?        & (LOGICAL)  & \\
\\
\verb?lgr?         & (LOGICAL)  & \\
\\
\verb?lstrfac?     & (LOGICAL)  & \\
\\
\verb?lmsd?        & (LOGICAL)  & \\
\\
\verb?lvacf?       & (LOGICAL)  & \\
\\
\end{tabular}

%###################################################################################################"
%                                         OPTTAG
%###################################################################################################"
\subsubsection{\&opttag}

Where : \verb?SUBROUTINE opt_init? in \verb?opt.f90?

\begin{tabular}{lcc}
\\
\verb?ncopt?     & (INTEGER) & \\

\\
\verb?nskipopt?  & (INTEGER) & \\

\\
\verb?nmaxopt?   & (INTEGER) & \\

\\
\verb?optalgo?   & (CHARACTER) & \\
\\
\end{tabular}

%###################################################################################################"
%                                         VIBTAG
%###################################################################################################"
\subsubsection{\&vibtag}

Where : \verb?SUBROUTINE vib_init? in \verb?vib.f90?

\begin{tabular}{lcc}
\\
\verb?ncvib?  & (INTEGER) & \\
\\
\verb?ngconf? & (INTEGER)  & \\
\\
\verb?lwrite_vectff? & (LOGICAL)  & \\
\\
\verb?imod? & (INTEGER)  & \\
\\
\verb?nkphon? & & \\
\\
\verb?resdos? & (DBLE) & \\
\\
\verb?omegamax? & (DBLE) & \\
\\
\verb?ksx,ksy,ksz? & & \\
\\
\verb?kfx,kfy,kfz? & & \\
\\
\end{tabular}

%###################################################################################################"
%                                         EFGTAG
%###################################################################################################"

\subsubsection{\&efgtag}

Where : \verb?SUBROUTINE efg_init? in \verb?efg.f90?

\begin{tabular}{lcc}
\\
\verb?lefgprintall? & (LOGICAL) & \\
\\
\verb?ncelldirect?  & (INTEGER) & \\
\\
\verb?ncefg?        & (INTEGER) & \\
\\
\verb?ncellewald?   & (INTEGER) & \\
\\
\verb?cutefg?  & (DBLE) & \\
\\
\verb?alphaES? & (DBLE) & \\
\\
\verb?resvzz?  & (DBLE) & \\
\\
\verb?reseta?  & (DBLE) & \\
\\
\verb?resu?    & (DBLE) & \\
\\
\verb?vzzmin?  & (DBLE) & \\ 
\\
\verb?umin?    & (DBLE) & \\
\\
\verb?ntcor? & & \\
\\
\end{tabular}

%###################################################################################################"
%                                         MSDTAG
%###################################################################################################"

\subsubsection{\&msdtag}

Where : \verb?SUBROUTINE msd_init? in \verb?msd.f90?

\begin{tabular}{lcc}
\\
\verb?tdifmax? & (DBLE) & \\
\\
\verb?nblock? & (DBLE) & \\
\\
\verb?ibmax? & (DBLE) & \\
\\
\end{tabular}

%###################################################################################################"
%                                         GRTAG
%###################################################################################################"

\subsubsection{\&grtag}

Where : \verb?SUBROUTINE control_init? in \verb?radial_distribution.f90?

\begin{tabular}{lcc}
\\
\verb?resg? & (DBLE) & resolution in the radial distribution function \\
\\
\end{tabular}

%###################################################################################################"
%                                         VACFTAG
%###################################################################################################"
\subsubsection{\&vacftag}

Where : \verb?SUBROUTINE vacf_init? in \verb?vacf.f90?


\begin{tabular}{lcc}
\\
\verb?tdifmax?  & (DBLE) & \\ 
\\
\verb?it0? & (DBLE) & \\
\\
\end{tabular}


%###################################################################################################"
%                                         OUTFILE DESCRIPTION
%###################################################################################################"
\subsection{Output files: description}


\section{Commands}

%DESCRIPTION:                                                 
% 
%  name   type   descritpion    default
%   TAG    (r)    something     (1.0D0 or ???)  
%                                                               
%  routine:
%  read control parameters in input files:
% 
%     CALCULATION:
% 
%        lstatic                 (l) static calculation -- only the potential energy of the input configuration  (.false.)
%        lnopbc                  (l) no periodic boundaries condition (.false.)
%        lbmlj                   (l) Binary Lennard-Jones calculation ( or simple LJ if  ntype=1) (.true.) 
%        calc                    (c) 'nve-vv', 'nve-lf', nve-and 'nvt'nh' or 'nvt-nhc2'    ('nve-vv')
% 
%                                         'nve-vv'     NVE ensemble verlet list algorithm  
%                                         'nve-lf'     NVE ensemble leap-frog algorithm    no thermostat
%                                         'nvt-and'    NVT+NVE Andersen thermostat 
%                                                       
%                                                      nuandersen    (r)  frequency of stochastic collision in the Andersen thermostat
%                                                                         probability of collision nuandersen*dt (???)
% 
%                                         'nvt-nh'     NVT ensemble Nove-Hoover thermostat 
%                                         'nvt-nhc2'   NVT ensemble Nose-Hoover chain two 2  
%  
%                                                      Qnosehoover  (r) nh mass parameter  (???)
% 
% 
%         (keys)                (i) like dl_poly 0, 1 or 2 ( pos, vel, force)
%        keyconfig              (i) pos input (??? WARN)
% 
%     MD:
% 
%        npas                   (i) number of time step (28)
%        dt                     (r) time step (???)            
%        temp                   (r) temperature  (???)
%        nequil                 (i) number of equilibration step (???)
%                                         berendsen : if tauberendsen= dt -> simple scaling 
%                                         scaling = ( 1 + (dt/tauberendsen)* ( temp/tempi -1)   ) ** 0.5
%        nequil_period          (i) rescaling period for the berendsen thermostat
%        cutoff                 (r) cutoff for the small range inter. (less than half the box WARNING)   
%        skindiff               (r) skindiff control verlet list update
%        largon                 (l) use argon parametrization (.false.)
%        
%       
%     CONFIG:
% 
%        natm                   (i) number of atom (???)
%        ntype                  (i) number of type if bmlj ntype is 1 or 2 ###### >2  (1)  
%        rho                    (r) density i'm not planning to simulate other cell than cubic cells ###### mass (???)
%        lfcc                   (l) start or not from fcc cell (.true.)
%        ncell                  (i) number of cell in that case natm = 4*(ncell)**3
% 
% 
%     TOOLS:
% 
%        lnovnlist             (l) do not use verlet list (.false.)
%        degauss               (c) 'boxmuller_basic', 'boxmuller_polar' or 'knuth'  (I said for fun aaaouh) (boxmuller_basic )
%                                   there are basically the same ( Is depending on the random number generator ??)
% 
%     OUTPUT:
%        
%        ltraj                 (l) save trajectory (.false.)
%        itraj_start           (i) in which step trajectory start to be saved (1)          
%        itraj_period          (i) save trajectory period (10000)
%        nprint                (i) print thermo information period to standard output(10)
%        fprint                (i) print thermo information period to OSZIFF file (10)
%        verbosity             (i) ###### well i don't know yet 
%        spas                  (i) ###### save configuration each spas: to have some backup for long run (1000)
%        nmax                  (i) ###### minimization sastry 


\section{Theory}
\cite{B-ALLEN_TILDESLEY}

%\section*{Acknowledgements} 

\bibliography{mdff}
\bibliographystyle{amsplain}

\end{document}
